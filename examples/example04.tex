% xelatex -synctex=1 -interaction=nonstopmode -shell-escape file.tex
\documentclass[a4paper,12pt, landscape]{report}
\usepackage[cm-default]{fontspec}
\usepackage[russian]{babel}

% полезные расширения, может быть полезным
\usepackage{xunicode, xltxtra}

%
%\usepackage{}
\usepackage{amsmath}
\usepackage{amssymb}
\usepackage{amsfonts}

\usepackage{multicol}
\usepackage{color}
\usepackage{longtable}

\usepackage[top=20mm, bottom=20mm, left=25mm, right=15mm]{geometry}
\setmainfont{Times New Roman} %CMU Serif}
\setsansfont{CMU Sans Serif}
\setmonofont{CMU Typewriter Text}
%
\author{Имя Фамилия}
\title{Название статьи}
\date{\today}
\definecolor{graylight}{rgb}{0.6, 0.6, 0.6}

\renewcommand{\le}{<=}

\begin{document}
\thispagestyle{empty}
%\maketitle % титульный лист; список исполнителей;
\begin{center}
%\textsc
\sc
ФЕДЕРАЛЬНОЕ  ГОСУДАРСТВЕННОЕ АВТОНОМНОЕ
ОБРАЗОВАТЕЛЬНОЕ УЧРЕЖДЕНИЕ ВЫСШЕГО ОБРАЗОВАНИЯ
«НАЦИОНАЛЬНЫЙ ИССЛЕДОВАТЕЛЬСКИЙ УНИВЕРСИТЕТ
«ВЫСШАЯ ШКОЛА ЭКОНОМИКИ»
\end{center}

\begin{center}
\bf Московский институт электроники и математики им.~А.Н.~Тихонова
\end{center}

\vspace{1cm}

\begin{center}
Фамилия Имя Отчество автора
\end{center}

\vspace{1cm}

\begin{center}
\bf НАЗВАНИЕ ТЕМЫ ВКР
\end{center}

\vspace{1cm}

\begin{center}
Выпускная квалификационная работа \par
по специальности 10.05.01 «Компьютерная безопасность» \par
студента образовательной программы специалитета
\end{center}

\vfill

\begin{multicols}{2}
\begin{flushleft}
~Студент\par\:\par
\begin{tabular}{c}
\underline{\hspace{8em}} \vspace{-2mm}\\
{\tiny {\color{graylight}подпись}}
\end{tabular}
\begin{tabular}{c}
\underline{\hspace{2em}Сидоров В.В.\hspace{1em}}\\
{\tiny{\color{graylight} ФИО}}
\end{tabular}
\end{flushleft}

\begin{flushright}
Научный руководитель

к.т.н., доцент
\end{flushright}

\begin{tabular}{c}
\underline{\hspace{8em}} \vspace{-1mm}\\
{\tiny {\color{graylight}подпись}}
\end{tabular}
\begin{tabular}{c}
\underline{\hspace{2em}Иванов И.И.\hspace{1em}} \vspace{-1mm}\\
{\tiny{\color{graylight} ФИО}}
\end{tabular}
\end{multicols}

\vfill
\begin{center}Москва, 2021 г.\end{center}
\newpage
\tableofcontents % содержание;

\setcounter{page}{2}
\renewcommand{\thepage}{строка номер \Asbuk{page}}

%\chapter{Реферат}
\chapter{Определения}\label{chapter::def}


Текстовая строка $\sin\pi = 0, \tan 0 = 0  $

\begin{gather*}
e = \lim_{n\to \infty} \left(1 + \frac{1}{n}\right)^n =
= \sum_{n=0}^\infty \frac{1}{n!}\\
e^x = \sum_{n=0}^\infty \frac{x^n}{n!}, 0 < |x| \le 1
\\
\int_{1}^{e^{12}} \frac{1}{x}dx = \ln x \big|_{1}^{e^{12}} =
\ln e^{12} = \ln 1 = 12.
\end{gather*}

\begin{multline}
e = \lim_{n\to \infty} \left(1 + \frac{1}{n}\right)^n =
= \sum_{n=0}^\infty \frac{1}{n!}\\
e^x = \sum_{n=0}^\infty \frac{x^n}{n!}, 0 < |x| \le 1
\\
\int_{1}^{e^{12}} \frac{1}{x}dx = \ln x \big|_{1}^{e^{12}} =
\ln e^{12} = \ln 1 = 12.
\end{multline}


Текстовая строка

Текстовая строка

$$
f(x) = x^3+1 \in \mathbb F_{q^n} = GF(q^n)
\mathbb N, \mathbb Q, \mathbb R, \mathbb C, \mathbb A = \mathbb Q(\sqrt[3]{17})
$$


Текстовая строка

%\begin{table}
%\begin{tabular}

$$ \pi \left(
\begin{array}{cccc}
1 & 2 & 3 & 4\\
3 & 1 & 2 & 4\\
\end{array}
\right)
$$

\begin{center}
\begin{longtable}{|c||c||c|}
\hline
Привет & мир & $\sin x$ \\
\hline
ривет & мир & $\sin x$ \\
\hline
\hline
Привет & мир & $\sin x$ \\
\hline
Привет & мир & $\sin x$ \\
Привет & мир & $\sin x$ \\
\hline
Привет & мир & $ \pi \left(
\begin{array}{cccc}
1 & 2 & 3 & 4\\
3 & 1 & 2 & 4\\
\end{array}
\right)
$
 \\
\hline
Привет & мир & $\sin x$ \\
\hline
Привет & мир & $\sin x$ \\
\hline
Привет & мир & $\sin x$ \\
\hline
Привет & мир & $\sin x$ \\
\hline
Привет & мир & $\sin x$ \\
\hline
Привет & мир & $\sin x$ \\
\hline
Привет & мир & $\sin x$ \\
\hline
Привет & мир & $\sin x$ \\
\hline
Привет & мир & $\sin x$ \\
\hline
Привет & мир & $\sin x$ \\
\hline
Привет & мир & $\sin x$ \\
\hline
Привет & мир & $\sin x$ \\
\hline
Привет & мир & $\sin x$ \\
\hline
Привет & мир & $\sin x$ \\
\hline
Привет & мир & $\sin x$ \\
\hline
Привет & мир & $\sin x$ \\
\hline
Привет & мир & $\sin x$ \\
\hline
Привет & мир & $\sin x$ \\
\hline
Привет & мир & $\sin x$ \\
\hline
Привет & мир & $\sin x$ \\
\hline
Привет & мир & $\sin x$ \\
\hline
Привет & мир & $\sin x$ \\
\hline
Привет & мир & $\sin x$ \\
\hline
Привет & мир & $\sin x$ \\
\hline
Привет & мир & $\sin x$ \\
\hline
Привет & мир & $\sin x$ \\
\hline
Привет & мир & $\sin x$ \\
\hline
Привет & мир & $\sin x$ \\
\hline
Привет & мир & $\sin x$ \\
\hline
Привет & мир & $\sin x$ \\
\hline
Привет & мир & $\sin x$ \\
\hline
Привет & мир & $\sin x$ \\
\hline
Привет & мир & $\sin x$ \\
\hline
Привет & мир & $\sin x$ \\
\hline
Привет & мир & $\sin x$ \\
\hline
Привет & мир & $\sin x$ \\
\hline
Привет & мир & $\sin x$ \\
\hline
Привет & мир & $\sin x$ \\
\hline
Привет & мир & $\sin x$ \\
\hline
Привет & мир & $\sin x$ \\
\hline
Привет & мир & $\sin x$ \\
\hline
Привет & мир & $\sin x$ \\
\hline
Привет & мир & $\sin x$ \\
\hline
Привет & мир & $\sin x$ \\
\hline
Привет & мир & $\sin x$ \\
\hline
Привет & мир & $\sin x$ \\
\hline
Привет & мир & $\sin x$ \\
\hline
Привет & мир & $\sin x$ \\
\hline
Привет & мир & $\sin x$ \\
\hline
Привет & мир & $\sin x$ \\
\hline
Привет & мир & $\sin x$ \\
\hline
Привет & мир & $\sin x$ \\
\hline
Привет & мир & $\sin x$ \\
\hline
Привет & мир & $\sin x$ \\
\hline
Привет & мир & $\sin x$ \\
\hline
Привет & мир & $\sin x$ \\
\hline
Привет & мир & $\sin x$ \\
\hline
Привет & мир & $\sin x$ \\
\hline
Привет & мир & $\sin x$ \\
\hline
Привет & мир & $\sin x$ \\
\hline
Привет & мир & $\sin x$ \\
\hline
Привет & мир & $\sin x$ \\
\hline
Привет & мир & $\sin x$ \\
\hline
Привет & мир & $\sin x$ \\
\hline
Привет & мир & $\sin x$ \\
\hline
Привет & мир & $\sin x$ \\
\hline
Привет & мир & $\sin x$ \\
\hline
Привет & мир & $\sin x$ \\
\hline
\end{longtable}
\end{center}

\chapter{Обозначения и сокращения}
\chapter{Введение}

\section{Раздел}

\begin{enumerate}
 \item Первая строка

\item Вторая строка, которая очень длинна
\begin{enumerate}
 \item\label{sss} Первая строка

\item Вторая строка, которая очень длинна

\end{enumerate}

\end{enumerate}

В разделе \ref{sss} fff

\subsection{Подраздел}\label{subsection::p1}

\subsubsection{Пункт}

В работе \cite{lenin123} сказано, что все будет хорошо ...

В работе \cite{lenin123} сказано, что все будет хорошо ...

В работе \cite{lenin123} сказано, что все будет хорошо ...

В работе \cite{lenin123} сказано, что все будет хорошо ...

В работе \cite{lenin123} сказано, что все будет хорошо ...

В работе \cite{lenin123} сказано, что все будет хорошо ...


\paragraph{Параграф} Текст

% основная часть

\chapter{Заключение}

в главе \ref{chapter::def}-й излагается ....

%\chapter{Список литература} % список использованных источников

\begin{thebibliography}{100}
\bibitem{lenin123} Ленин (Ульянов) В.И. Собрание сочинений. Том 2. -- 1972.
\end{thebibliography}

\texttt{}
\tt
%находится на странице \pageref{chapter::def}

\chapter{Приложение}

В работе \cite{lenin123} сказано, что все будет хорошо ...

В работе \cite{lenin123} сказано, что все будет хорошо ...

\end{document}
