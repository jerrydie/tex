% xelatex -synctex=1 -interaction=nonstopmode -shell-escape file.tex
\documentclass[a4paper,12pt, landscape]{report}
\usepackage[cm-default]{fontspec}
\usepackage[russian]{babel}

% полезные расширения, может быть полезным
\usepackage{xunicode, xltxtra}

%
%\usepackage{}
\usepackage{amsmath}
\usepackage{amssymb}
\usepackage{amsfonts}

\usepackage{multicol}
\usepackage{color}
\usepackage{longtable}

\usepackage[top=20mm, bottom=20mm, left=25mm, right=15mm]{geometry}
\setmainfont{Times New Roman} %CMU Serif}
\setsansfont{CMU Sans Serif}
\setmonofont{CMU Typewriter Text}
%
\author{Имя Фамилия}
\title{Название статьи}
\date{\today}
\definecolor{graylight}{rgb}{0.6, 0.6, 0.6}

\renewcommand{\le}{<=}

\usepackage{tikz}
\usepackage{pgfplots}
\usetikzlibrary{arrows, automata, graphs}
\usepackage{floatflt}

\begin{document}

%\begin{tikzpicture}
%%\draw (-5,-5) grid (5,5);
%%\draw (-2,-2) rectangle (5,5);
%
%\draw [red, ultra thick] (0,0) circle [radius=0.1];
%%\draw (3,3) node [below] {$\int_0^\infty xdx $};
%
%\draw[->] (0.2,0.3) -- (2,1);
%\draw[->] (2,1) -- (3.1,4.3);
%\draw[->] (3.1,4.3) -- (4,2) -- (5,1.7);
%
%\draw (7,1) circle [radius=0.7];
%\draw (7,1) node (a1) [draw] {$\texttt{A}_1$};
%\draw (9,2) node (a2) [draw] {\begin{tabular}{c}$\texttt{A}_2$\\$\sqrt{3}$\end{tabular}};
%\draw[->] (7.7765,1.25) -- (a2);
%\end{tikzpicture}


%\begin{figure}[!h]
%\begin{center}

\begin{floatingfigure}[rflt]{7cm}
\begin{tikzpicture}[->,>=stealth',shorten >=1pt,auto, node distance=1.5cm, semithick]
\tikzstyle{every state}=[]

 \node[state] (C2) {С2};
 \node[state] (C1) [above left of=C2] {С1};
 \node[state] (C23) [above right of=C2] {С3};
 \node[state] (C13) [below right of=C23] {С13};
 \node[state] (C7) [below of=C2] {c7};
 \node[state] (C8) [above left of=C7] {c8};
 \node[state] (C14) [below of=C7] {c14};
 \node[state] (C27) [below left of=C1] {27};

 \path (C1) edge [bend left] (C2)
       (C2) edge (C13)
       (C13) edge [bend left] (C7)
       (C7) edge (C14);
 \path (C23) edge [bend right] (C2);
 \path (C2) edge (C7);
 \path (C8) edge (C7);
 \path (C27) edge [bend right] (C14);
 \path (C2) edge [bend right] (C27);
\end{tikzpicture}
\end{floatingfigure}

Предыдущий текст

Некий текст
Некий текст в конце

%\caption{Красивый рисунок}
%\label{fig:property-c14}
%\end{center}
%\end{figure}



\begin{tikzpicture}
\begin{axis}[
	title = Exponenta,
	xlabel = {$x$},
	ylabel = {$y$},
	minor tick num = 2
]
\addplot[blue] coordinates { (0.2,0.3) (2,1) (3.1,4.3) (4,2) (5,1.7) };
\end{axis}
\end{tikzpicture}


Еще текст

%\begin{tikzpicture}
%  \begin{loglogaxis}
%    \addplot coordinates {
%       (769,   1.6227e-04)
%       (1793,  4.4425e-05)
%       (4097,  1.2071e-05)
%       (9217,  3.2610e-06)
%       (2.2e5, 2.1E-6)
%       (1e6,   0.00003341)
%       (2.3e7, 0.00131415) };
%  \end{loglogaxis}
%\end{tikzpicture}


%%%%%%%%%%%%%%%%%%%%%%%%%%%%%%%%%%%%%%%%%%%%%%%%%%%%%%%%%%%%%%%%%%%%%%%%%%%%%%%%%%%%%%%%%%%
%%%%%%%%%%%%%%%%%%%%%%%%%%%%%%%%%%%%%%%%%%%%%%%%%%%%%%%%%%%%%%%%%%%%%%%%%%%%%%%%%%%%%%%%%%%
\begin{center}
  \begin{tikzpicture}
  \coordinate (O) at (0,0);
  \coordinate (A) at (0,3);
  \def\r{1} % radius
  \def\c{1.4} % center
  \coordinate (C) at (\c, \r);


  \draw[-latex] (O) -- (A) node[anchor=south] {$y$};
  \draw[-latex] (O) -- (2.6*pi,0) node[anchor=west] {$x$};

  \draw[red,domain=-0.5*pi:2.5*pi,samples=50, line width=1]
       plot ({2^\x - sin(\x r)},{1 - cos(\x r)});

%  \draw[blue, line width=1] (C) circle (\r);
%  \draw[] (C) circle (\r);
%
%  % coordinate x
%  \def\x{0.4} % coordinate x
%  \def\y{0.83} % coordinate y
%  \def\xa{0.3} % coordinate x for arc left
%  \def\ya{1.2} % coordinate y for arc left
%  \coordinate (X) at (\x, 0 );
%  \coordinate (Y) at (0, \y );
%  \coordinate (XY) at (\x, \y );
%
%  \node[anchor=north] at (X) {$x$} ;
%
%  % draw center of circle
%  \draw[fill=blue] (C) circle (1pt);
%
%  % draw radius of the circle
%  \draw[] (C) -- node[anchor=south] {\; $a$} (XY);
%
%  % bottom of circle, radius to the bottom
%  \coordinate (B) at (\c, 0);
%  \draw[] (C) -- (B) node[anchor=north] {$a \, \theta$};
%
%  % projections of point XY
%  \draw[dotted] (XY) -- (X);
%  \draw[dotted] (XY) -- (Y) node[anchor=east, xshift=1mm] {$\quad y$};
%
%  % arc theta
%  % start arc
%  \coordinate (S) at (\c, 0.4);
%  \draw[->] (S) arc (-90:-165:0.6);
%  \node[xshift=-2mm, yshift=-2mm] at (C) {\scriptsize $\theta$};
%
%  % arc above
%  \coordinate (AA) at (\xa, \ya);
%  \draw[-latex, rotate=25] (AA) arc (-220:-260:1.3);
%
%  % arc below
%  \def\xb{2.5} % coordinate x for arc bottom
%  \def\yb{0.8} % coordinate y for arc bottom
%  \coordinate (AB) at (\xb, \yb);
%  \draw[-latex, rotate=-10] (AB) arc (-5:-45:1.3);
%
%
%
%  % XY dot
%  \draw[fill=black] (XY) circle (1pt);
%
%
%  % top label
%  \coordinate (T) at (pi, 2);
%  \node[anchor=south] at (T)  {$(\pi a, 2 a )$} ;
%  \draw[fill=black] (T) circle (1pt);
%
%  % equations
%  \coordinate (E) at ( 4,1.2);
%  \coordinate (F) at ( 4,0.9);
%  \node[] at (E) {\scriptsize $x=a(\theta - \sin \theta)$};
%  \node[] at (F) {\scriptsize $y=a(1 - \cos \theta)$};
%
%  % label 2pi a
%  \coordinate (TPA) at (2*pi, 0);
%  \node[anchor=north] at (TPA) {$2 \pi a$};
%

  \end{tikzpicture}
\end{center}
%%%%%%%%%%%%%%%%%%%%%%%%%%%%%%%%%%%%%%%%%%%%%%%%%%%%%%%%%%%%%%%%%%%%%%%%%%%%%%%%%%%%%%%%%%%
%%%%%%%%%%%%%%%%%%%%%%%%%%%%%%%%%%%%%%%%%%%%%%%%%%%%%%%%%%%%%%%%%%%%%%%%%%%%%%%%%%%%%%%%%%%

\end{document}
