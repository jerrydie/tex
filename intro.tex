\Introduction

Целью работы является демонстрация навыков подготовки электронных документов в системе компьютерной верстки документов latex. Выполнены требования к подготавливаемому документу:

\begin{itemize}
\item титульный лист по ГОСТ 7.32-2001;
\item общий объем документа не менее 18 страниц;
\item наличие разделов документа, включая ненумерованные (введение, обозначения и т.п.),
\item наличие формул (строчные и выключенные),
\item наличие ссылок (по документу и внешних),
\item наличие таблиц, изображений, графиков и т.п.,
\item наличие списка литературы и ссылок на него по тексту документа,
\item определение собственных команд, упрощающих процесс набора документа,
\item отсутствие орфографических ошибок, наличие смысла в подготовленном документе,
обоснование, для чего документ подготовлен,
\item краткая информация о сборке документа и используемых шрифтах, размерах (шриф-
тов, полей и т.п.),
\item информация о том, какие стилевые пакеты применяли и для какой-цели.
\end{itemize}

Содержательно документ подготовлен для демонстрации студентам, проходящим курс теории псевдослучайных генераторов, хода решения типовых задач из контрольноых работ, для некоторых типов задач приведено решение для всех вариантов входных параметров. Первыми разбираются задачи из контрольной работы 2, так как часть из них входит в контрольную работу 1, далее разбираются задачи из первой контрольной, не вошедшие во вторую.


Обозначения и определения "---
MAX, 
\Abbrev{ЛРП}{Линейная рекурентная последовательность}
API 
\Abbrev{API}{application programming interface ""--- внешний интерфейс взаимодействия с приложением}
с обратным прокси.
\Define{Обратный прокси}{тип прокси-сервера, который ретранслирует}
