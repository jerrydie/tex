%% Преамбула TeX-файла

% 1. Стиль и язык
\documentclass[utf8x, 14pt]{G7-32} % Стиль (по умолчанию будет 14pt)

% Остальные стандартные настройки убраны в preamble.inc.tex.
\sloppy

% Настройки стиля ГОСТ 7-32

% Добавляем гипертекстовое оглавление в PDF
\usepackage[
bookmarks=true, colorlinks=true, unicode=true,
urlcolor=black,linkcolor=black, anchorcolor=black,
citecolor=black, menucolor=black, filecolor=black, urlcolor=blue,
]{hyperref}
\AfterHyperrefFix

\usepackage{microtype}% пакет для микротипографии, под pdflatex хорошо улучшает читаемость

% Тире могут быть невидимы в Adobe Reader
\ifInvisibleDashes
\MakeDashesBold
\fi

\usepackage{graphicx}   % Пакет для включения рисунков
\usepackage{tikz}
\graphicspath{ {./images/} }

\geometry{right=20mm}
\geometry{left=30mm}
\geometry{bottom=20mm}
\geometry{ignorefoot}


\setlength{\parskip}{1ex} % разрыв между абзацами

% Упрощение написания ссылки
\newcommand\case[1]{По Случаю #1 из Теоремы 2.6 из \cite{hse:Teoria_Gener}}





% Полезные макросы листингов.
%\include{macros.inc}

% Стиль титульного листа и заголовки
% Титульная страница, по умолчанию ГОСТ 7.32-2001,  
 
\NirOrgLongName{Правительство Российской Федерации\par
Федеральное государственное автономное образовательное\\
учреждение высшего образования\\
"Национальный исследовательский университет\\
"Высшая школа экономики"\\
Московский институт электроники и математики им. А.Н. Тихонова\\
Департамент прикладной математики
} 


\NirReportName{Методические материалы}
\NirAbout{По дисциплине \par "Теория псевдослучайных генераторов"} 


\Nir{Разбор практических задач с теоретическим обоснованием}

\NirSubject{Типовые задачи контрольных работ}                                
\NirCode{}

\NirManager{Преподаватель дисциплины}{М.И. Рожков }
\NirIsp{Руководитель создания документа}{А.Ю. Нестеренко}

\NirYear{2021}
\NirTown{Москва}



\begin{document}

\frontmatter % выключает нумерацию ВСЕГО; здесь начинаются ненумерованные главы: реферат, введение, глоссарий, сокращения и прочее.

\maketitle %создает титульную страницу


\begin{executors}
\personalSignature{Выполнил студент}{Щеглова П.Н.}
\end{executors}


%\listoffigures                         % Список рисунков

%\listoftables                          % Список таблиц

%\NormRefs % Нормативные ссылки 
% Команды \breakingbeforechapters и \nonbreakingbeforechapters
% управляют разрывом страницы перед главами.
% По-умолчанию страница разрывается.

% \nobreakingbeforechapters
% \breakingbeforechapters

\tableofcontents

\printnomenclature % Автоматический список сокращений

\Introduction

Целью работы является демонстрация навыков подготовки электронных документов в системе компьютерной верстки документов latex. Выполнены требования к подготавливаемому документу:

\begin{itemize}
\item титульный лист по ГОСТ 7.32-2001;
\item общий объем документа не менее 18 страниц;
\item наличие разделов документа, включая ненумерованные (введение, обозначения и т.п.);
\item наличие формул (строчные и выключенные);
\item наличие ссылок (по документу и внешних);
\item наличие таблиц, изображений и т.п.;
\item наличие списка литературы и ссылок на него по тексту документа;
\item определение собственных команд, упрощающих процесс набора документа;
\item отсутствие орфографических ошибок, наличие смысла в подготовленном документе,
обоснование, для чего документ подготовлен;
\item краткая информация о сборке документа и используемых шрифтах, размерах (шрифтов, полей и т.п.);
\item информация о том, какие стилевые пакеты применяли и для какой цели.
\end{itemize}

Содержательно документ подготовлен для демонстрации студентам, проходящим курс теории псевдослучайных генераторов, хода решения типовых задач из контрольноых работ, для некоторых типов задач приведено решение для всех вариантов входных параметров. Первыми разбираются общие задачи для двух контрольных, затем уникальные для первой контрольной задачи и для второй. Теоретические обоснования даны в виде ссылок на соответствующие утверждения из источников и частично продублированы в виде утверждений. 


\chapter{Сборка документа}
\label{cha:design}
\section{Архитектура всячины}

\subsection{Протестируем подпункт}
\subsubsection{А теперь подподпункт}


\paragraph{Проверка} параграфа. Вроде работает.
\paragraph{Вторая проверка} параграфа. Опять работает.


\begin{itemize}
\item Это список с <<палочками>>.
\item Хотя он и по ГОСТ, но\dots
\end{itemize}



\mainmatter % это включает нумерацию глав и секций в документе ниже

\chapter{Задачи контрольных работ 2 и 1}
\section{Разбор примеров задач контрольной работы 2}
\subsection{Задача 1}
Найти минимальный многочлен $m(x)$ для ЛРП с х.м. $f(x)\in\{x^5+x+1, x^5+x^4+1, x^5+x^2+1, x^5+x^3+1\}\in F_2[x]$ и вектором начального состояния: $$s(0)\in\{(11111), (11011), (11101), (11100), (10110), (00001), (00011), (00111), (01111)\}$$

По утверждению 1.6.2 любой характеристический многочлен ЛРП кратен минимальному многочлену. Тогда достаточно проверить делители характеристической функции.

\begin{enumerate}
    \item $x^5+x+1$
    
    Так как корней нет ($0$ или $1$ не подходят), делители следует искать срели многочленов степени $2,3$, причем неприводимых. Таких 2 степени в нашем поле только одно - $1 + x + x^2$, проверим делимость простым делением уголком - остаток от деления равен $0$, получаем, что $f(x) = (x^2+x+1)(x^3+x^2+1)$. $x^3+x^2+1$ - в свою очередь не имеет корней и не делится на $1 + x + x^2$, а значит неприводим. Так как ЛРП ненулевая (начальный вектор ненулевой), делитель $1$ также не является минимальным многочленом по определению. Следовательно минимальный многочлен $m(x)\in \{x^2+x+1, x^3+x^2+1, x^5+x+1\}$.
    
    Проверим многочлен $(x^2+x+1)$:  многочлен показывает, что каждое следующее число последовательности получается из двух предыдущих как линейная комбинация с коэффициентами как в многочлене, то есть если первые числа последовательности $1,1$, то ЛРП соответствует вектору начального состояния $(1, 1, 0, 1, 1)$, если первые числа последовательности $0,1$, то $(0, 1, 1, 0, 1)$, и по такому же принципу подходит вектор начального состояния $(1,0,1,1,0)$. Таким образом, для полученных векторов $m(x) = (x^2+x+1)$.  ( P.S. теория говорит нам, что проверять совпадение последовательности необходимо и достаточно лишь в начальном векторе, далее последовательность будет вычисляться в соответствии с характеристической функцией, минимальный многочлен должен совпадать таким образом и иметь минимальную подходящую степень, поэтому оставшиеся многочлены-претенденты не подходят).
    
    Проверим многочлен $(x^3+x^2+1)$:  многочлен показывает, что каждое следующее число последовательности получается из трех предыдущих как линейная комбинация с коэффициентами как в многочлене, то есть , в нашем случае сумма первого и третьего задает четвертое, тогда если первые числа последовательности $1,1,1$, то ЛРП соответствует вектору начального состояния $(1, 1, 1, 0, 1)$, если первые числа последовательности $1,1,0$, то $(1, 1, 0, 1, 0)$, и по такому же принципу подходят вектора начального состояния $(1,0,1,0,0), (1,0,0,1,1), (0,1,1,1,0), (0,1,0,0,1), (0,0,1,1,1)$. Таким образом, для полученных векторов $m(x) = (x^3+x^2+1)$.
    
    Для всех остальных состояний кроме $(0,0,0,0,0)$, минимальный многочлен будет совпадать с характеристической функцией.
    
    \item $x^5+x^4+1$
    
    Так как корней нет ($0$ или $1$ не подходят), делители следует искать срели многочленов степени $2,3$, причем неприводимых. Таких 2 степени в нашем поле только одно - $1 + x + x^2$, проверим делимость простым делением уголком - остаток от деления равен $0$, получаем, что $f(x) = (x^2+x+1)(x^3+x+1)$. $x^3+x+1$ - в свою очередь не имеет корней и не делится на $1 + x + x^2$, а значит неприводим. Так как ЛРП ненулевая (начальный вектор ненулевой), делитель $1$ также не является минимальным многочленом по определению. Следовательно минимальный многочлен $m(x)\in \{x^2+x+1, x^3+x+1, x^5+x^4+1\}$.
    
    Проверим многочлен $(x^2+x+1)$:  многочлен показывает, что каждое следующее число последовательности получается из двух предыдущих как линейная комбинация с коэффициентами как в многочлене, то есть если первые числа последовательности $1,1$, то ЛРП соответствует вектору начального состояния $(1, 1, 0, 1, 1)$, если первые числа последовательности $0,1$, то $(0, 1, 1, 0, 1)$, и по такому же принципу подходит вектор начального состояния $(1,0,1,1,0)$. Таким образом, для полученных векторов $m(x) = (x^2+x+1)$.  ( P.S. теория говорит нам, что проверять совпадение последовательности необходимо и достаточно лишь в начальном векторе, далее последовательность будет вычисляться в соответствии с характеристической функцией, минимальный многочлен должен совпадать таким образом и иметь минимальную подходящую степень, поэтому оставшиеся многочлены-претенденты не подходят).
    
    Проверим многочлен $(x^3+x+1)$:  многочлен показывает, что каждое следующее число последовательности получается из трех предыдущих как линейная комбинация с коэффициентами как в многочлене, то есть , в нашем случае сумма первого и второго задает четвертое, тогда если первые числа последовательности $1,1,1$, то ЛРП соответствует вектору начального состояния $(1, 1, 1, 0, 0)$, если первые числа последовательности $1,1,0$, то $(1, 1, 0, 0, 1)$, и по такому же принципу подходят вектора начального состояния $(1,0,1,1,1), (1,0,0,1,0), (0,1,1,1,0), (0,1,0,1,1), (0,0,1,0,1)$. Таким образом, для полученных векторов $m(x) = (x^3+x+1)$.
    
    Для всех остальных состояний кроме $(0,0,0,0,0)$, минимальный многочлен будет совпадать с характеристической функцией.
    
    \item $x^5+x^2+1$
    
    Так как корней нет ($0$ или $1$ не подходят), делители следует искать срели многочленов степени $2,3$, причем неприводимых. Таких 2 степени в нашем поле только одно - $1 + x + x^2$, проверим делимость простым делением уголком - остаток от деления равен $1$, получаем, что $f(x)$ неприводим. Так как ЛРП ненулевая (начальный вектор ненулевой), делитель $1$ также не является минимальным многочленом по определению. Следовательно минимальный многочлен $m(x) = x^5+x^4+1$ для любого ненулевого начального вектора.
    
    \item $ x^5+x^3+1$
    
    Так как корней нет ($0$ или $1$ не подходят), делители следует искать срели многочленов степени $2,3$, причем неприводимых. Таких 2 степени в нашем поле только одно - $1 + x + x^2$, проверим делимость простым делением уголком - остаток от деления равен $x+1$, получаем, что $f(x)$ неприводим. Так как ЛРП ненулевая (начальный вектор ненулевой), делитель $1$ также не является минимальным многочленом по определению. Следовательно минимальный многочлен $m(x) = x^5+x^4+1$ для любого ненулевого начального вектора.    

    
\end{enumerate}

\subsection{Задача 2}
Для каких $m(x)\in\{m_1(x)=x^2+x+1, m_2(x)=x^3+x+1, m_3(x)=x^3+x^2+1, m_4(x)=x+1\}$ в семействе $S(f)$, $f(x)\in \{x^5+x+1, x^5+x^4+1, x^4+x^2+1, x^4+1, x^3+1\}, f\in F_2[x]$, существует ЛРП, минимальный многочлен которой равен $m(x)$?

Если многочлен является неприводимым делителем функции $f$ из семейства $S(f)$, то в этом семействе обязательно найдется ЛРП, для которой данный многочлен является минимальным. 

$f(x) = $

\begin{enumerate}
    \item $x^5+x+1$
    
     Так как корней нет ($0$ или $1$ не подходят), делители следует искать срели многочленов степени $2,3$, причем неприводимых. Таких 2 степени в нашем поле только одно - $1 + x + x^2$, проверим делимость простым делением уголком - остаток от деления равен $0$, получаем, что $f(x) = (x^2+x+1)(x^3+x^2+1)$. $x^3+x^2+1$ - в свою очередь не имеет корней и не делится на $1 + x + x^2$, а значит неприводим. Тогда для многочленов $m_1(x)=x^2+x+1, m_3(x) = x^3+x^2+1$ существуют ЛРП из данного семейства, для которых они  являются минимальным. Например, см. Разбор примеров задач контрольной работы, Задача 1, пункт 1.
     
     \item $x^5+x^4+1$
     
     Так как корней нет ($0$ или $1$ не подходят), делители следует искать срели многочленов степени $2,3$, причем неприводимых. Таких 2 степени в нашем поле только одно - $1 + x + x^2$, проверим делимость простым делением уголком - остаток от деления равен $0$, получаем, что $f(x) = (x^2+x+1)(x^3+x+1)$. $x^3+x+1$ - в свою очередь не имеет корней и не делится на $1 + x + x^2$, а значит неприводим. Тогда для многочленов $m_1(x)=x^2+x+1, m_2(x)=x^3+x+1$ существуют ЛРП из данного семейства, для которых они  являются минимальным.  Например, см. Разбор примеров задач контрольной работы, Задача 1, пункт 2.
     
     \item $x^4+x^2+1$
     Так как корней нет ($0$ или $1$ не подходят), делители следует искать срели многочленов степени $2$, причем неприводимых. Таких 2 степени в нашем поле только одно - $1 + x + x^2$, проверим делимость простым делением уголком - остаток от деления равен $0$, получаем, что $f(x) = x^4+x^2+1 = (x^2+x+1)^2$. Тогда для многочлена $m_1(x)=x^2+x+1$ существуют ЛРП из данного семейства, для которых он  является минимальным.  Например, см. Разбор примеров задач контрольной работы, Задача 1, пункт 1 или 2.
     
     \item $x^4+1$
     $f(x)$ приводим: $(x+1)^4$. Тогда для многочлена $m_4(x)=x+1$ существуют ЛРП из данного семейства, для которых он  является минимальным. 
     
     \item $x^3+1$
     $f(x)$ приводим: $(x+1)(x^2+x+1)$. Тогда для многочлена $m_1(x)=x^2+x+1, m_4(x)=x+1$ существуют ЛРП из данного семейства, для которых он  является минимальным. 
     
\end{enumerate}

\subsection{Задача 3}
Для каких $r\in\{1,2,3,…,30\}$ в семействе $S(f)$ ЛРП с х.м. $f(x)\in\{(x^2+x+1)^8, (x+1)^{10}, (x^3+x+1)^4, (x^2+x+1)^2(x+1)^4, (x+1)^{12}, (x^3+x^2+1)^3\}, f\in F_2[x]$, существует ЛРП с минимальным периодом, равным $r$?

Минимальный период ЛРП равен периоду его минимального многочлена, следовательно достаточно для каждой х.ф. вычислить периоды всех ее делителей:

\begin{enumerate}
    \item $(x^2+x+1)^8$, его делитель, многочлен $x^2+x+1$ не имеет корней и неприводим, его период $r = 2^2 -1 = 3$, так как второй делитель этого числа - 1 - не подходит, в связи с тем, что период многочлена степени выше 1 должен быть больше 1. Далее, наименьшее такое $t$, что $2^t \geq 8$, $t=3$. В этом случае по Случаю 2 период ненулевой последовательности из $S(f)$ будет лежать в множестве $\{3, 6, 12, 24\}$. Тогда, так как для любого делителя х.ф. в семействе ЛРП найдется такая, для которой он будет минимальным многочленом, тогда для всех представленных в множестве вариантов найдется такая ЛРП, для которой минимальный период принадлежит множеству $r \in \{1,3,6,12,24 \}$. 
    
    $x^3 (\mod x^2+x+1) = (x+1)x = x^2+x = 1, r = 3$
    
    $x^3 (\mod (x^2+x+1)^2) = x^3 \neq 1$
    
    $x^6 (\mod (x^2+x+1)^2 = x^4+x^2+1) = (x^2+1)x^2 = x^4+x^2 = 1, r = 6$
    
    $x^3 (\mod (x^2+x+1)^3) = x^3 \neq 1$
    
    $x^6 (\mod (x^2+x+1)^3 = 1 + x+ x^3+x^5+ x^6) = 1 + x+ x^3+x^5 \neq 1$, тогда $r = 12$
    
    $x^{12} (\mod (x^2+x+1)^8 ) = x^{12} \mod 1 + x^8 + x^{16} = x^{12} \neq 1$
    
    $x^{24} \mod 1 + x^8 + x^{16} = (1 + x^8) x^8 = x^8 + x^{16} = 1, r = 24$
    
    \item $(x+1)^{10}$, его делитель, многочлен $x+1$ не имеет корней и неприводим, его период $r = 2^1 -1 = 1$. Далее, наименьшее такое $t$, что $2^t \geq 10$, $t=4$. В этом случае период ненулевой последовательности из $S(f)$ будет лежать в множестве $\{1, 2, 4, 8, 16\}$. Тогда, так как для любого делителя х.ф. в семействе ЛРП найдется такая, для которой он будет минимальным многочленом, тогда для всех представленных в множестве вариантов найдется такая ЛРП, для которой минимальный период принадлежит множеству $r \in \{1, 2, 4, 8, 16\}$.
    
    $x^1 \mod x+1 = 1, r = 1$
    
    $x^2 \mod x^2+1 = 1, r = 2$
    
    $x^4 \mod x^3 + x^2 + x +1 = (x^2 + x + 1) x = 1, r = 4$
    
    $x^8 \mod x^4+1 = 1, r = 8$
    
    $x^8 \mod x^{10} +x^8 + x^2 + 1 = x^8 \neq 1$
    
    $x^{16} \mod x^{10} +x^8 + x^2 + 1  = (x^8 + x^2 + 1) x^6 = x^{14} + x^8 + x^6 = (x^8 + x^2 + 1) x^4 + x^8 + x^6 = x^{12} + x^8 + x^4$
    $ = (x^8 + x^2 + 1) x^2 + x^8 + x^4 = x^{10} + x^8 + x^2 = 1, r = 16$
    
    \item $(x^3+x+1)^4$, его делитель, многочлен $x^3+x+1$ не имеет корней и неприводим, его период $r = 2^3 -1 = 7$, так как второй делитель этого числа - 1 - не подходит, в связи с тем, что период многочлена степени выше 1 должен быть больше 1. Далее, наименьшее такое $t$, что $2^t \geq 4$, $t=2$. В этом случае период ненулевой последовательности из $S(f)$ будет лежать в множестве $\{7, 14, 28\}$. Тогда, так как для любого делителя х.ф. в семействе ЛРП найдется такая, для которой он будет минимальным многочленом, тогда для всех представленных в множестве вариантов найдется такая ЛРП, для которой минимальный период принадлежит множеству $r \in \{1, 7, 14, 28 \}$.
    
    $x^7 \mod x^3+x+1 = (x+1)(x+1)x = x^3 + x = 1, r = 7$
    
    $x^7 \mod (x^3+x+1)^2 = x^7 \mod 1+x^2+x^6 = x+x^3 \neq 1$
    
    $x^{14} \mod 1+x^2+x^6 = (1+x^2)^2 x^2 = x^2 + x^6 = 1, r = 14$
    
    $x^{14} \mod (x^3+x+1)^4 = x^{14} \mod 1 + x^4 + x^{12} = (1+x^4) x^2 = x^2 + x^6 \neq 1$
    
    $x^{28} \mod (x^3+x+1)^4 = x^{28} \mod 1 + x^4 + x^{12} = (1+x^4)^2 x^4 = (1 + x^8) x^4 = x^4 + x^{12} = 1, r = 28$
    
    \item $(x^2+x+1)^2(x+1)^4$, его делитель, многочлен $x^2+x+1$ не имеет корней и неприводим, его период $r = 2^2 -1 = 3$, так как второй делитель этого числа - 1 - не подходит, в связи с тем, что период многочлена степени выше 1 должен быть больше 1; его делитель, многочлен $x+1$ не имеет корней и неприводим, его период $r = 2^1 -1 = 1$. Далее, наименьшее такое $t$, что $2^t \geq 2$, $t=1$, наименьшее такое $t$, что $2^t \geq 4$, $t=2$. В этом случае период ненулевой последовательности из $S(f)$ будет лежать в множестве $\{3, 6; 1, 2, 4; 12\}$. Тогда, так как для любого делителя х.ф. в семействе ЛРП найдется такая, для которой он будет минимальным многочленом, тогда для всех представленных в множестве вариантов найдется такая ЛРП, для которой минимальный период принадлежит множеству $r \in \{1,2,3,4,6,12 \}$.
    \item $(x+1)^{12}$, его делитель, многочлен $x+1$ не имеет корней и неприводим, его период $r = 2^1 -1 = 1$. Далее, наименьшее такое $t$, что $2^t \geq 12$, $t=4$. В этом случае период ненулевой последовательности из $S(f)$ будет лежать в множестве $\{1, 2, 4, 8, 16\}$. Тогда, так как для любого делителя х.ф. в семействе ЛРП найдется такая, для которой он будет минимальным многочленом, тогда для всех представленных в множестве вариантов найдется такая ЛРП, для которой минимальный период принадлежит множеству $r \in \{1, 2, 4, 8, 16\}$.
    
    $x^1 \mod x+1 = 1, r = 1$
    
    $x^2 \mod x^2+1 = 1, r = 2$
    
    $x^4 \mod x^3 + x^2 + x +1 = (x^2 + x + 1) x = 1, r = 4$
    
    $x^8 \mod x^4+1 = 1, r = 8$
    
    $x^8 \mod x^12 +x^8 + x^4 + 1 = x^8 \neq 1$
    
    $x^{16} \mod x^{12} +x^8 + x^4 + 1 = (x^8 + x^4 + 1) x^4 = x^{12} + x^8 + x^4 = 1, r = 16 $
    
    \item $(x^3+x^2+1)^3$, его делитель, многочлен $x^3+x^2+1$ не имеет корней и неприводим, его период $r = 2^3 -1 = 7$, так как второй делитель этого числа - 1 - не подходит, в связи с тем, что период многочлена степени выше 1 должен быть больше 1. Далее, наименьшее такое $t$, что $2^t \geq 3$, $t=2$. В этом случае по Случаю 2 период ненулевой последовательности из $S(f)$ будет лежать в множестве $\{7, 14, 28\}$. Тогда, так как для любого делителя х.ф. в семействе ЛРП найдется такая, для которой он будет минимальным многочленом, тогда для всех представленных в множестве вариантов найдется такая ЛРП, для которой минимальный период принадлежит множеству $r \in \{1, 7, 14, 28\}$. 
    
    $x^7 \mod x^3+x^2+1 = (x^2+1)(x^2+1)x = x^5 + x = (x^2+1)x^2 +x = x^4 + x^2 + x = x^3 + x^2 = 1, r = 7$
    
    $x^7 \mod (x^3+x^2+1)^2 = x^7 \mod 1+x^4+x^6 = x+x^5 \neq 1$
    
    $x^{14} \mod 1+x^4+x^6 = (1+x^4)^2 x^2 = x^2 + x^10 = (1+x^4)x^4 + x^2 = (1+x^4)x^2 + x^4 + x^2 = 1, r = 14$
    
    $x^{14} \mod (x^3+x^2+1)^3 = x^{14} \mod 1 + x^2 + x^3 + x^4 + x^7 + x^8 + x^{9} = (1 + x^2 + x^3 + x^4 + x^7 + x^8) x^5 = x^5 + x^7 + x^8 + x^9 + x^{12} + x^{13} = 1 + x^2 + x^3 + x^4 + x^5 + (1+x) (x^3 + x^5 + x^6 + x^7 + x^10 + x^11) ... = 1+x+x^5+x^7 \neq 1$
    
    $x^{28} \mod (x^3+x^2+1)^3 = x^{28} \mod 1 + x^2 + x^3 + x^4 + x^7 + x^8 + x^{9} = ... = 1, r = 28$ Без вариантов.
    
\end{enumerate}

\subsection{Задача 4}
Найти первые три бита (нумерация координат слева направо) вектора 57–го состояния ЛРП с х.м. $f(x)=(x^3+x^2+1)^4 \in F_2[x]$ и вектором начального (ненулевого) состояния $s(0)\in (F2)^{12}, s(0)\in\{ (101100…0), (110100…0), (111100…0), (011000…0), (1000…0), $ $(0100…0), (101000…0), (100100…0)\}$.

Делитель характеристической функции, многочлен $x^3+x^2+1$ не имеет корней и неприводим, его период $r = 2^3 -1 = 7$, так как второй делитель этого числа - 1 - не подходит, в связи с тем, что период многочлена степени выше 1 должен быть больше 1. Далее, наименьшее такое $t$, что $2^t \geq 4$, $t=2$. В этом случае по Случаю 2 период ненулевой последовательности из $S(f)$ будет лежать в множестве $\{7, 14, 28\}$. 

$x^{14} \mod (x^3+x^2+1)^4 = x^{14} \mod 1 + x^8 + x^{12} = (1+x^8) x^2 = x^2 + x^10 \neq 1$

$x^{28} \mod (x^3+x^2+1)^4 = x^{28} \mod 1 + x^8 + x^{12} = (1+x^8)^2 x^4 = (1 + x^{16}) x^4 = x^4 + x^{20} =$

$(1+x^8)x^8 +x^4 = (1+x^8)x^4 + x^8 + x^4 = 1, r = 28$

Тогда 57 состояние ЛРП $s(57) = s(28*2 + 1) = s(1)$. Тогда достаточно выработать одну новую координату и сдвинуть начало вектора, по виду х.ф. видно, что каждый следующий бит получается из 12 предыдущих как сумма первого бита ($x^{12}$) и пятого ($x^8$). Тогда $s(57) = s(1) \in\{ (01100…01), (10100…01), (11100…01), (11000…01), $ $(000…01), (100…01), (01000…01), (00100…01)\}$. Тогда ответ - первые три бита получившейся последовательности.

\subsection{Задача 5}
Найти минимальный период суммы двух ЛРП с х.м. $f_1(x)\in\{x^3+x^2+1, x^3+x+1\}, f_1\in F_2[x]$, и $f_2(x)\in\{x^2+x+1, x^2+1\}, f_2\in F_2[x]$, при условии, что начальный вектор первой ЛРП равен $s_1 \in\{(010), (110), (111), (101), (011), (001), (100)\}$, а второй равен $s_2\in\{(11), (10), (01)\}$.
\begin{enumerate}
    \item $f_1(x) = x^3+x^2+1, f_1\in F_2[x]$, $f_2(x) = x^2+x+1, f_2\in F_2[x]$
    
     Многочлены $x^3+x^2+1$ и $x^2+x+1$ не имеют корней, а значит неприводимы. Характеристический многочлен суммы двух ЛРП $f(x) = (x^3+x^2+1)(x^2+x+1)$ 5 степени, следовательно вектор начального состояния для данной ЛРП имеет длину 5. Для установленного варианта начальных векторов выработаем последовательность до 5 знаков и просуммируем, чтобы получить начальный вектор суммы ЛРП: $v_1 \in\{(01001), (11010), (11101), (10100), (01110), (00111), (10011)\}$, $v_2\in\{(11011), (10110), (01101)\}$, следовательно, если $s_2 = (11)$, то $s \in \{(10010), (00001),$ $(00110), (01111), (10101), (11100), (01000)\}$, если $s_2 = (10)$, то $s \in\{(11111), (01100)$, $(01011), (00010), (01000), (10001), (00101)\}$, если $s_2 = (01)$, то $s \in\{(00100), (10111)$, $(10000), (11001), (00011), (01010), (11110)\}$.
     
     В зависимости от начального вектора минимальным многочленом может быть $m(x)\in\{x^2+x+1, x^3+x^2+1, (x^3+x^2+1)(x^2+x+1)\}$, 1 не подходит, так как начальный вектор ненулевой. Определяем для полученного начального состояния, начиная с многочлена-претендента меньшей степени, может ли он вырабатывать данную последовательность и получаем, что минимальный период может быть равен $r \in \{3, 7, 21 \}$, будучи периодом найденного минимального многочлена.
     
     \item $f_1(x) = x^3+x^2+1, f_1\in F_2[x]$, $f_2(x) = x^2+1, f_2\in F_2[x]$
    
     Многочлен $x^3+x^2+1$ не имеет корней, а значит неприводим, $x^2+1 = (x+1)^2$. Характеристический многочлен суммы двух ЛРП $f(x) = (x^3+x^2+1)(x^2+1)$ 5 степени, следовательно вектор начального состояния для данной ЛРП имеет длину 5. Для установленного варианта начальных векторов выработаем последовательность до 5 знаков и просуммируем, чтобы получить начальный вектор суммы ЛРП: считаем аналогично предудыщему варианту.
     
     В зависимости от начального вектора минимальным многочленом может быть $m(x)\in\{x+1, x^2+1, x^3+x^2+1, (x^3+x^2+1)(x+1), (x^3+x^2+1)(x^2+1)\}$, 1 не подходит, так как начальный вектор ненулевой. Определяем для полученного начального состояния, начиная с многочлена-претендента меньшей степени, может ли он вырабатывать данную последовательность и получаем, что минимальный период может быть равен $r \in \{1, 2, 7, 7, 14 \}$, будучи периодом найденного минимального многочлена.
     
     \item $f_1(x) = x^3+x+1, f_1\in F_2[x]$, $f_2(x) = x^2+x+1, f_2\in F_2[x]$
    
     Многочлен $x^3+x+1$ и $x^2+x+1$ не имеют корней, а значит неприводимы. Характеристический многочлен суммы двух ЛРП $f(x) = (x^3+x+1)(x^2+x+1)$ 5 степени, следовательно вектор начального состояния для данной ЛРП имеет длину 5. Для установленного варианта начальных векторов выработаем последовательность до 5 знаков и просуммируем, чтобы получить начальный вектор суммы ЛРП: считаем аналогично предудыщему варианту.
     
     В зависимости от начального вектора минимальным многочленом может быть $m(x)\in\{x^2+x+1, x^3+x+1, (x^3+x+1)(x^2+x+1)\}$, 1 не подходит, так как начальный вектор ненулевой. Определяем для полученного начального состояния, начиная с многочлена-претендента меньшей степени, может ли он вырабатывать данную последовательность и получаем, что минимальный период может быть равен $r \in \{3, 7, 21 \}$, будучи периодом найденного минимального многочлена.
     
     \item $f_1(x) = x^3+x+1, f_1\in F_2[x]$, $f_2(x) = x^2+1, f_2\in F_2[x]$
    
     Многочлен $x^3+x+1$ не имеет корней, а значит неприводим, $x^2+1 = (x+1)^2$. Характеристический многочлен суммы двух ЛРП $f(x) = (x^3+x+1)(x^2+1)$ 5 степени, следовательно вектор начального состояния для данной ЛРП имеет длину 5. Для установленного варианта начальных векторов выработаем последовательность до 5 знаков и просуммируем, чтобы получить начальный вектор суммы ЛРП: считаем аналогично предудыщему варианту.
     
     В зависимости от начального вектора минимальным многочленом может быть $m(x)\in\{x+1, x^2+1, x^3+x+1, (x^3+x+1)(x+1), (x^3+x+1)(x^2+1)\}$, 1 не подходит, так как начальный вектор ненулевой. Определяем для полученного начального состояния, начиная с многочлена-претендента меньшей степени, может ли он вырабатывать данную последовательность и получаем, что минимальный период может быть равен $r \in \{1, 2, 7, 7, 14 \}$, будучи периодом найденного минимального многочлена.
\end{enumerate}

\subsection{Задача 6}
Найти минимальный многочлен суммы двух ЛРП с х.м. $f_1(x)\in\{x^3+x^2+1, x^3+x+1\}, f_1\in F_2[x]$, и $f_2(x)\in\{x^3+x+1, x^3+1\}, f_2\in F_2[x]$, при условии, что начальный вектор первой ЛРП равен $s_1 \in \{(010), (111), (110), (001), (100),(101),(011)\}$, а второй равен $s_2 \in\{(101), (111), (010), (001), (011), (100), (110)\}$

\begin{enumerate}
    \item $f_1(x) = x^3+x^2+1, f_1\in F_2[x]$, $f_2(x) = x^3+x+1, f_2\in F_2[x]$
    
     Многочлены $x^3+x^2+1$ и $x^3+x+1$ не имеют корней, а значит неприводимы. Характеристический многочлен суммы двух ЛРП $f(x) = (x^3+x^2+1)(x^3+x+1)$ 6 степени, следовательно вектор начального состояния для данной ЛРП имеет длину 6. Для установленного варианта начальных векторов выработаем последовательность до 6 знаков и просуммируем, чтобы получить начальный вектор суммы ЛРП: вычисляем аналогично Задаче 5.
     
     В зависимости от начального вектора минимальным многочленом может быть $m(x)\in\{x^3+x+1, x^3+x^2+1, (x^3+x^2+1)(x^3+x+1)\}$, 1 не подходит, так как начальный вектор ненулевой. Определяем для полученного начального состояния, начиная с многочлена-претендента меньшей степени, может ли он вырабатывать данную последовательность и получаем минимальный многочлен.
     
     \item $f_1(x) = x^3+x^2+1, f_1\in F_2[x]$, $f_2(x) = x^3+1, f_2\in F_2[x]$
    
     Многочлены $x^3+x^2+1$  не имеет корней, а значит неприводим, $x^3+1 = (x+1)(x^2+x+1)$. Характеристический многочлен суммы двух ЛРП $f(x) = (x^3+x^2+1)(x^3+1)$ 6 степени, следовательно вектор начального состояния для данной ЛРП имеет длину 6. Для установленного варианта начальных векторов выработаем последовательность до 6 знаков и просуммируем, чтобы получить начальный вектор суммы ЛРП: вычисляем аналогично Задаче 5.
     
     В зависимости от начального вектора минимальным многочленом может быть $m(x)\in\{x+1, x^2+x+1, x^3+1, x^3+x^2+1, (x^3+x^2+1)(x+1), (x^3+x^2+1)(x^2+x+1), (x^3+x^2+1)(x^3+1)\}$, 1 не подходит, так как начальный вектор ненулевой. Определяем для полученного начального состояния, начиная с многочлена-претендента меньшей степени, может ли он вырабатывать данную последовательность и получаем минимальный многочлен.
     
     \item $f_1(x) = x^3+x+1, f_1\in F_2[x]$, $f_2(x) = x^3+x+1, f_2\in F_2[x]$
    
     Многочлен $x^3+x+1$ не имеет корней, а значит неприводим. Характеристический многочлен суммы двух ЛРП $f(x) = (x^3+x+1)^2$ 6 степени, следовательно вектор начального состояния для данной ЛРП имеет длину 6. Для установленного варианта начальных векторов выработаем последовательность до 6 знаков и просуммируем, чтобы получить начальный вектор суммы ЛРП: вычисляем аналогично Задаче 5.
     
     В зависимости от начального вектора минимальным многочленом может быть $m(x)\in\{x^3+x+1, (x^3+x+1)^2\}$, 1 не подходит, так как начальный вектор ненулевой. Определяем для полученного начального состояния, начиная с многочлена-претендента меньшей степени, может ли он вырабатывать данную последовательность и получаем минимальный многочлен.
     
     \item $f_1(x) = x^3+x+1, f_1\in F_2[x]$, $f_2(x) = x^3+1, f_2\in F_2[x]$
    
     Многочлены $x^3+x+1$  не имеет корней, а значит неприводим, $x^3+1 = (x+1)(x^2+x+1)$. Характеристический многочлен суммы двух ЛРП $f(x) = (x^3+x+1)(x^3+1)$ 6 степени, следовательно вектор начального состояния для данной ЛРП имеет длину 6. Для установленного варианта начальных векторов выработаем последовательность до 6 знаков и просуммируем, чтобы получить начальный вектор суммы ЛРП: вычисляем аналогично Задаче 5.
     
     В зависимости от начального вектора минимальным многочленом может быть $m(x)\in\{x+1, x^2+x+1, x^3+1, x^3+x+1, (x^3+x+1)(x+1), (x^3+x+1)(x^2+x+1), (x^3+x+1)(x^3+1)\}$, 1 не подходит, так как начальный вектор ненулевой. Определяем для полученного начального состояния, начиная с многочлена-претендента меньшей степени, может ли он вырабатывать данную последовательность и получаем минимальный многочлен.
\end{enumerate}

\subsection{Задача 7}
Заданы две ЛРП с х.м. $f_1(x)=x^2+x+1 \in F_2[x], f_2(x)=x^2+1\in F_2[x]$. Начальные состояния указанных ЛРП равны соответственно $ (11, 11), (11, 10), (11, 01), (10, 11),$ $ (01, 11), (01, 01), $ $(01, 10), (10, 10), (10, 01) $. Найти минимальный период их произведения.

Первая характеристический многочлен $f_1(x)=x^2+x+1$ непримводим, тогда он совпадает с минимальным многочленом и его период - минимальный период первой ЛРП $\omega_1 = 2^{deg(f_1(x))=2} -1 = 3$. Второй характеристический многочлен приводим $f_2(x)=x^2+1 = (x+1)^2$, тогда в зависимости от начального состояния период данной ЛРП равен $\omega_2 = 1$ (для вектора $(11)$) или $\omega_2 = 2$. Произведение ЛРП - также ЛРП, и ее период (не обязательно минимальный) равен $\omega = $НОК$(\omega_1,\omega_2) = 3$ или НОК$(\omega_1,\omega_2) = 6$.

Рассмотрим оба принципиальных варианта:
\begin{enumerate}
    \item (11, 11)
    $\omega = $НОК$(\omega_1,\omega_2) = 3; x^w+1$ - х.м.
    
    Выработаем 3 знак - (110, 111), тогда начальный вектор произведения ЛРП (110). $x^3+1 = (x+1)(x^2+x+1)$, и как видно из начального вектора $(x^2+x+1)$ - минимальный многочлен, тогда минимальный период ЛРП равен 3.
    \item (11,10)
    $\omega = $НОК$(\omega_1,\omega_2) = 6; x^w+1$ - х.м.
    
    Выработаем до 6 знака - (110110, 101010), тогда начальный вектор произведения ЛРП (100010). $x^6+1 = (x+1)^2(x^2+x+1)^2$, и как видно из начального вектора $(x+1)^2(x^2+x+1)$ - минимальный многочлен, тогда минимальный период ЛРП равен 6.
\end{enumerate}

\subsection{Задача 8}
Укажите функциональную связь между выходным знаком с номером $m = \{5,6,7,8,9,10\}$ и координатами начального состояния $x=(x_1,x_2,x_3)$ фильтрующего генератора, вырабатывающего выходную последовательность по закону: 

$\gamma_1 = f(x)$

$\gamma_2=f(\delta_L(x))$

...

$\gamma_j=f(\delta_{L^{j-1}}(x))$

...

$f(x)=f(x_1,x_2,x_3)\in\{x_1 x_3+x_2, x_1 x_2+x_3\}$;

$\delta_L(x)=\delta_L(x_1,x_2,x_3)=(x_2,x_3,L(x)), L(x)=L(x_1,x_2,x_3)= x_1+x_2$

\begin{enumerate}
    \item $f(x)=x_1 x_3 + x_2$
    
    $\gamma_1= x_1 x_3 +x_2$
    
    $\gamma_2 = x_2 (x_1 + x_2) + x_3 = x_1 x_2 + x_2 + x_3$
    
    $\gamma_3 = x_2 x_3 + x_3 + x_1 + x_2$
    
    $\gamma_4 = x_3 (x_1 + x_2) + x_1 + x_2 + x_2 + x_3 = x_1 x_3 + x_2 x_3 + x_1 + x_3$
    
    $\gamma_5 = x_2 (x_1 + x_2) + x_3 (x_1 + x_2) + x_2 + x_1 + x_2 = x_1 x_2 + x_1 x_3 + x_2 x_3 + x_2 +x_1$
    
    $\gamma_6 = x_2 x_3 + x_2 (x_1 + x_2) + x_3 (x_1 + x_2) + x_2 + x_3 = x_1 x_2 + x_1 x_3 + x_3$
    
    $\gamma_7 = x_2 x_3 + x_2 (x_1 + x_2) + x_1 + x_2 = x_1 x_2 + x_2 x_3 + x_1$
    
    $\gamma_8 = x_2 x_3 + x_3 (x_1 + x_2) + x_2 = x_1 x_3 + x_2$
    
    Зациклилось.
    
    
    \item $f(x)=x_1 x_2 + x_3$
    
    $\gamma_1 = x_1 x_2 + x_3$
    
    $\gamma_2 = x_2 x_3 + x_1 + x_2$
    
    $\gamma_3 = x_3 (x_1 + x_2) + x_2 + x_3 = x_1 x_3 + x_2 x_3 + x_2 + x_3$
    
    $\gamma_4 = x_2 (x_1 + x_2) + x_3 (x_1 + x_2) + x_3 + x_1 + x_2 = x_1 x_2 + x_1 x_3 + x_2 x_3 + x_1 + x_3$
    
    $\gamma_5 = x_2 x_3 + x_2 (x_1 + x_2) + x_3 (x_1 + x_2) +x_2 + x_1 + x_2 = x_1 x_2 + x_1 x_3 + x_1 + x_2$
    
    $\gamma_6 = x_2 x_3 + x_2 (x_1 + x_2) + x_2 + x_3 = x_1 x_2 + x_2 x_3 + x_3$
    
    $\gamma_7 = x_2 x_3 + x_3 (x_1 + x_2) + x_1 + x_2 = x_1 x_3 + x_1 + x_2$
    
    $\gamma_8 = x_2 (x_1 + x_2) + x_2 + x_3 = x_1 x_2 + x_3$
    
    Зациклилось.
\end{enumerate}


\subsection{Задача 9}
Найти вероятность выходной 3-гаммы $\gamma = (\gamma_1,\gamma_2,\gamma_3) \in \{ (000), (001), (010),$ $ (100), (110), (011),$ $(101), (111) \}$ при условии равновероятного входа для фильтрующей схемы, задаваемой функцией $f=f(x_1,x_2,x_3)\in \{x_1 x_3+x_2, x_1 x_3+x_1+x_2+x_3, x_1 x_3+x_2+1\}$, $\gamma_j = f(x_j, x_{j+1}, x_{j+2}), j=1,2,3$.

$x_1 // x_2 // x_3 // x_4 // x_5 // (\gamma_1,\gamma_2,\gamma_3) = (x_1 x_3 + x_2, x_2 x_4 + x_3, x_3 x_5 + x_4) // (x_1 x_3+x_1+x_2+x_3, x_2 x_4+x_2+x_3+x_4, x_3 x_5+x_3+x_4+x_5) // (x_1 x_3+x_2+1, x_2 x_4+x_3+1, x_3 x_5+x_4+1)$

$0 // 0 // 0 // 0 // 0 // (0,0,0) // (0,0,0) // (1,1,1)$

$0 // 0 // 0 // 0 // 1 // (0,0,0) // (0,0,1) // (1,1,1)$

$0 // 0 // 0 // 1 // 0 // (0,0,1) // (0,1,1) // (1,1,0)$

$0 // 0 // 0 // 1 // 1 // (0,0,1) // (0,1,0) // (1,1,0)$

$0 // 0 // 1 // 0 // 0 // (0,1,0) // (1,1,1) // (1,0,1)$

$0 // 0 // 1 // 0 // 1 // (0,1,1) // (1,1,1) // (1,0,0)$

$0 // 0 // 1 // 1 // 0 // (0,1,1) // (1,0,0) // (1,0,0)$

$0 // 0 // 1 // 1 // 1 // (0,1,0) // (1,0,0) // (1,0,1)$

$0 // 1 // 0 // 0 // 0 // (1,0,0) // (1,1,0) // (0,1,1)$

$0 // 1 // 0 // 0 // 1 // (1,0,0) // (1,1,1) // (0,1,1)$

$0 // 1 // 0 // 1 // 0 // (1,1,1) // (1,1,1) // (0,0,0)$

$0 // 1 // 0 // 1 // 1 // (1,1,1) // (1,1,0) // (0,0,0)$

$0 // 1 // 1 // 0 // 0 // (1,1,0) // (0,0,1) // (0,0,1)$

$0 // 1 // 1 // 0 // 1 // (1,1,1) // (0,0,1) // (0,0,0)$

$0 // 1 // 1 // 1 // 0 // (1,0,1) // (0,0,0) // (0,1,0)$

$0 // 1 // 1 // 1 // 1 // (1,0,0) // (0,0,0) // (0,1,1)$

$1 // 0 // 0 // 0 // 0 // (0,0,0) // (1,0,0) // (1,1,1)$

$1 // 0 // 0 // 0 // 1 // (0,0,0) // (1,0,1) // (1,1,1)$

$1 // 0 // 0 // 1 // 0 // (0,0,1) // (1,1,1) // (1,1,0)$

$1 // 0 // 0 // 1 // 1 // (0,0,1) // (1,1,0) // (1,1,0)$

$1 // 0 // 1 // 0 // 0 // (1,1,0) // (1,1,1) // (0,0,1)$

$1 // 0 // 1 // 0 // 1 // (1,1,1) // (1,1,1) // (0,0,0)$

$1 // 0 // 1 // 1 // 0 // (1,1,1) // (1,0,0) // (0,0,0)$

$1 // 0 // 1 // 1 // 1 // (1,1,0) // (1,0,0) // (0,0,1)$

$1 // 1 // 0 // 0 // 0 // (1,0,0) // (0,1,0) // (0,1,1)$

$1 // 1 // 0 // 0 // 1 // (1,0,0) // (0,1,1) // (0,1,1)$

$1 // 1 // 0 // 1 // 0 // (1,1,1) // (0,1,1) // (0,0,0)$

$1 // 1 // 0 // 1 // 1 // (1,1,1) // (0,1,0) // (0,0,0)$

$1 // 1 // 1 // 0 // 0 // (0,1,0) // (0,0,1) // (1,0,1)$

$1 // 1 // 1 // 0 // 1 // (0,1,1) // (0,0,1) // (1,0,0)$

$1 // 1 // 1 // 1 // 0 // (0,0,1) // (0,0,0) // (1,1,0)$

$1 // 1 // 1 // 1 // 1 // (0,0,0) // (0,0,0) // (1,1,1)$

Тогда вероятности комбинаций гаммы следующие:

$(\gamma_1,\gamma_2,\gamma_3)// (x_1 x_3 + x_2, x_2 x_4 + x_3, x_3 x_5 + x_4) // (x_1 x_3+x_1+x_2+x_3, x_2 x_4+x_2+x_3+x_4, x_3 x_5+x_3+x_4+x_5) // (x_1 x_3+x_2+1, x_2 x_4+x_3+1, x_3 x_5+x_4+1)$

$0//0//0//\dfrac{5}{2^5}//\dfrac{5}{2^5}//\dfrac{7}{2^5}$

$0//0//1//\dfrac{5}{2^5}//\dfrac{5}{2^5}//\dfrac{3}{2^5}$

$0//1//0//\dfrac{3}{2^5}//\dfrac{3}{2^5}//\dfrac{1}{2^5}$

$0//1//1//\dfrac{3}{2^5}//\dfrac{3}{2^5}//\dfrac{5}{2^5}$

$1//0//0//\dfrac{5}{2^5}//\dfrac{5}{2^5}//\dfrac{3}{2^5}$

$1//0//1//\dfrac{1}{2^5}//\dfrac{1}{2^5}//\dfrac{3}{2^5}$

$1//1//0//\dfrac{3}{2^5}//\dfrac{3}{2^5}//\dfrac{5}{2^5}$

$1//1//1//\dfrac{7}{2^5}//\dfrac{7}{2^5}//\dfrac{5}{2^5}$

\subsection{Задача 10}
Найти период нелинейной рекурренты $x_1,x_2,...$, вырабатываемой из начального вектора-состояния $(x_1,x_2,...,x_n)= (x_1,x_2,x_3,x_4)=(1010)$, $f=x_1+L(x_2,...,x_n)+(x_2+1)(x_3+1)\cdot\cdot\cdot(x_n+1) = x_1+ L(x_2,x_3,x_4)+(x_2+1)(x_3+1)(x_4+1)$, $L(x_2,x_3,x_4)\in\{x_2,x_3,x_4,x_2+x_3,x_2+x_4,x_3+x_4,x_2+x_3+x_4\}$.

\begin{enumerate}
    \item $L(x_2,x_3,x_4) = x_2$
    
    $f(x) = x_1 + x_2 + \overline{x_2}\cdot \overline{x_3}\cdot \overline{x_4}$
    
    $x_5 = x_1 + x_2 + \overline{x_2}\cdot \overline{x_3}\cdot \overline{x_4} = 1 + 0 + 1\cdot0\cdot1=1$
    
    $x_6 = x_2 + x_3 +  \overline{x_3}\cdot \overline{x_4}\cdot \overline{x_5} = 0 + 1 + 0\cdot1\cdot0 = 1$
    
    $x_7 = x_3 + x_4 +  \overline{x_4}\cdot \overline{x_5}\cdot \overline{x_6} = 1 + 0 + 1\cdot0\cdot0 = 1$
    
    $x_8 = x_4 + x_5 +  \overline{x_5}\cdot \overline{x_6}\cdot \overline{x_7} = 0 + 1 + 0\cdot0\cdot0 = 1$
    
    $x_9 = x_5 + x_6 +  \overline{x_6}\cdot \overline{x_7}\cdot \overline{x_8} = 1 + 1 + 0\cdot0\cdot0 = 0$
    
    $x_{10} = x_6 + x_7 +  \overline{x_7}\cdot \overline{x_8}\cdot \overline{x_9} = 1 + 1 + 0\cdot0\cdot1 = 0$
    
    $x_{11} = x_7 + x_8 +  \overline{x_8}\cdot \overline{x_9}\cdot \overline{x_{10}} = 1 + 1 + 0\cdot1\cdot1 = 0$
    
    $x_{12} = x_8 + x_9 +  \overline{x_9}\cdot \overline{x_{10}}\cdot \overline{x_{11}} = 1 + 0 + 1\cdot1\cdot1 = 0$
    
    $x_{13} = x_9 + x_{10} +  \overline{x_{10}}\cdot \overline{x_{11}}\cdot \overline{x_{12}} = 0 + 0 + 1\cdot1\cdot1 = 1$
    
    $x_{14} = x_{10} + x_{11} +  \overline{x_{11}}\cdot \overline{x_{12}}\cdot \overline{x_{13}} = 0 + 0 + 1\cdot1\cdot0 = 0$
    
    $x_{15} = x_{11} + x_{12} +  \overline{x_{12}}\cdot \overline{x_{13}}\cdot \overline{x_{14}} = 0 + 0 + 1\cdot0\cdot1 = 0$
    
    $x_{16} = x_{12} + x_{13} +  \overline{x_{13}}\cdot \overline{x_{14}}\cdot \overline{x_{15}} = 0 + 1 + 0\cdot1\cdot1 = 1$
    
    $x_{17} = x_{13} + x_{14} +  \overline{x_{14}}\cdot \overline{x_{15}}\cdot \overline{x_{16}} = 1 + 0 + 1\cdot1\cdot0 = 1$
    
    $x_{18} = x_{14} + x_{15} +  \overline{x_{15}}\cdot \overline{x_{16}}\cdot \overline{x_{17}} = 0 + 0 + 1\cdot0\cdot0 = 0$
    
    $x_{19} = x_{15} + x_{16} +  \overline{x_{16}}\cdot \overline{x_{17}}\cdot \overline{x_{18}} = 0 + 1 + 0\cdot0\cdot1 = 1$
    
    $x_{20} = x_{16} + x_{17} +  \overline{x_{17}}\cdot \overline{x_{18}}\cdot \overline{x_{19}} = 1 + 1 + 0\cdot1\cdot0 = 0$
    
    Таким образом, $(x_1, x_2, x_3, x_4) = (x_{17}, x_{18}, x_{19}, x_{20}) \Rightarrow r = 16$
    
\end{enumerate}


\section{Невошедшее из контрольной работы №1}

\subsection{Задача 7}
Найти период многочлена:
\begin{enumerate}
    \item $f(x)=x^4+x^3+x^2+x+1; F_2[x]$
    
    Корней у многочлена нет ($1,-1$ не подходят), поэтому необходимо рассмотреть неприводимые многочлены 2 степени. Таких в нашем поле только одно - $1 + x + x^2$, проверим делимость простым делением уголком - остаток от деления равен $x+1$. Таким образом, многочлен $f(x)$ неприводим, тогда в соответствии со случаем 1 период должен быть делителем числа $ q^m -1 = 2^4 - 1 = 15$. Рассмотрим числа $\{3, 5, 15\}$. $x^3 \mod x^4+x^3+x^2+x+1 = x^3 \neq 1$, $x^5 \mod x^4+x^3+x^2+x+1 = x(x^3+x^2+x+1) = 1$, следовательно $r= 5$.

    \item $f(x)=2x+1; Z_3[x]$
    
    Многочлен первой степени, поэтому он неприводим, но тогда период многочлена может быть равен $1$ или $3^1-1 = 2$. $x -1 (\mod{2x + 1}) = 2\dfrac{x + 2}{2} (\mod{2x + 1}) = 2(2x + 1) (\mod{2x + 1}) = 0$, следовательно период $r = 1$.
    
    \item $f(x)=x+1; Z_3[x]$
    
    Многочлен первой степени, поэтому он неприводим, но тогда период многочлена может быть равен $1$ или $3^1-1 = 2$. $x - 1 (\mod{x + 1}) = (x + 2) (\mod{x + 1}) = 1 \neq 0$, следовательно период $r \neq 1$. Таким образом, тогда в соответствии со случаем 1 период многочлена $r = q^m -1 = 3^1 -1 = 2$.
    
    \item $f(x)=x^2+x+1; Z_3[x]$
    
    $f(x) = x^2 + x + 1 = x^2 -2x +1 = (x - 1)^2$. Таким образом, в соответстви со случаем 2, так как $x - 1 (\mod{x - 1}) = 0$, $e = r(x-1) = 1$, $p = 3$, $t$ – наименьшее натуральное число, при котором $p^t\geq n$, тогда $t = 1$. Тогда период многочлена может быть ${1, 3}$. Период многочлена степени выше 1 должен быть больше 1, поэтому в данном примере $r = 3$.

    \item $f(x)=x^2+x+2; Z_3[x]$
    
    Так как корней у многочлена нет ($1,2,3$ не подходят), он неприводим. Период многочлена степени выше 1 должен быть больше 1, тогда в соответствии со случаем 1 период многочлена должен быть делителем числа $q^m -1 = 3^2 -1 = 8$. Рассмотрим числа $\{2, 4, 8\}$. $x^2 \mod x^2 + x +2 = - x - 2 \neq 1$,  $x^4 \mod x^2 + x +2 = (2x + 1)^2 = x^2 + x + 1 = - 1 \neq 1$. Следовательно, $r=8$.

    \item $f(x)=x^2+2x+2; Z_3[x]$
    
    Так как корней у многочлена нет ($1,2,3$ не подходят), он неприводим. Период многочлена степени выше 1 должен быть больше 1, тогда в соответствии со случаем 1 период многочлена должен быть делителем числа $q^m -1 = 3^2 -1 = 8$. Рассмотрим числа $\{2, 4, 8\}$. $x^2 \mod x^2 + 2x +2 = x +1 \neq 1$,  $x^4 \mod x^2 + 2x +2 = (x +1)^2 = x^2 + 2x + 1 = - 1 \neq 1$. Следовательно, $r=8$.

    \item $f(x)=x^2+1; Z_3[x]$
    
    Так как корней у многочлена нет ($1,2,3$ не подходят), он неприводим. Период многочлена степени выше 1 должен быть больше 1, тогда в соответствии со случаем 1 период многочлена должен быть делителем числа $q^m -1 = 3^2 -1 = 8$. Рассмотрим числа $\{2, 4, 8\}$. $x^2 \mod x^2 + 1 = -1 \neq 1$,  $x^4 \mod x^2 + 1 = (-1)^2 = 1$. Следовательно, $r=4$.

    \item $f(x)=x^3+1; Z_3[x]$
    
    Так как у многочлена есть корень $2$, делим уголком на $x-2 = x+1$ и получаем $f(x) = (x+1) (x^2 + 2x +1) = (x+1)^3$. Таким образом в соответствии со случаем 2, так как $x - 1 (\mod{x + 1}) = 1 \neq 0$, $e = r(x-1) = 3^1 - 1 = 2$, $p = 3$, $t$ – наименьшее натуральное число, при котором $p^t\geq n$, тогда $t = 1$. Тогда период многочлена может быть ${2, 6}$. Проверим для $r = 2: x^2 (\mod x^3 + 1) = x^2 \neq 1$, тогда в данном примере $r = 6$.

    \item $f(x)=x^3+x+1; Z_3[x]$
    
    Так как у многочлена есть корень $1$, делим уголком на $x-1$ и получаем $f(x) = (x-1) (x^2 + x +2)$. Из примера 5. видим, что $(x^2 + x +2)$ неприводим и его период $r(x^2 + x +2) = 8$. Период $x -1$, так как $x -1 (\mod{x - 1}) = 0$, равен $1$. Тогда в соответствии со случаем 3: $r =$НОК$(r(x-1), r(x^2 + x +2)) = $НОК$(1, 8) = 8$.
    

    \item $f(x)=x^3+x+2; Z_3[x]$
    Так как у многочлена есть корень $2$, делим уголком на $x-2 = x+1$ и получаем $f(x) = (x+1) (x^2 + x +2)$. Из примера 5. видим, что $(x^2 + x +2)$ неприводим и его период $r(x^2 + x +2) = 8$. Период $x + 1$, из примера 3., равен $2$. Тогда в соответствии со случаем 3: $r =$НОК$(r(x+1), r(x^2 + x +2)) = $НОК$(2, 8) = 8$.
    
\end{enumerate}

\subsection{Задача 8}
Укажите все примитивные многочлены из множества:


P.S. Многочлен называется примитивным, если он неприводим и имеет максимальный период $q^m - 1$.

\begin{enumerate}
    \item $f(x)\in\{x+1, x^2+1, x^2+x+1, x^3+1\}, (f(x)\in F_2[x])$
    
    $x \mod x+1 = 1$, следовательно $ r = 1 = q^m -1 = 2^1 -1$, тогда многочлен имеет максимальный период и неприводим, тк имеет степень 1. Примитивен.
    
    $x^2 + 1 = (x+1)^2$, приводим. Не примитивен.
    
    $ x^2+x+1$ непримводим, тк не имеет корней. В соответствии со случаем 1, так как период многочлена степени выше 1 должен быть больше 1, $r = q^m -1 = 2^2 - 1 = 3$, тк 3 - простое число. Примитивен.
    
    $x^3 + 1 = (x+1)(x^2 - x + 1)$ приводим. Не примитивен.

    \item $f(x)\in\{x^2+1, x^3+1, x^3+x+1\}, (f(x)\in F_2[x])$
    
    $x^2 + 1 = (x+1)^2$, приводим. Не примитивен.
    
    $x^3 + 1 = (x+1)(x^2 - x + 1)$ приводим. Не примитивен.
    
    $x^3+x+1$ неприводим, тк не имеет корней. В соответствии со случаем 1, так как период многочлена степени выше 1 должен быть больше 1, $r = q^m -1 = 2^3 - 1 = 7$, тк 7 - простое число. Примитивен.

    \item $f(x)\in\{x^3+1, x^3+x^2+1, x^4+x^2+1\}, (f(x)\in F_2[x])$
    
    $x^3 + 1 = (x+1)(x^2 - x + 1)$ приводим. Не примитивен.
    
    $x^3+x^2+1$ не имеет корней, неприводим. В соответствии со случаем 1, так как период многочлена степени выше 1 должен быть больше 1, $r = q^m -1 = 2^3 - 1 = 7$, тк 7 - простое число. Примитивен.
    
    $ x^4+x^2+1$ не имеет корней, приводим: $ x^4+x^2+1 = (x^2+x+1)^2$. Не примитивен.

    \item $f(x)\in\{x^5+1, x^5+x+1, x^4+x^2+x+1\}, (f(x)\in F_2[x])$
    
    $x^5+1$ имеет корень 1, приводим. Не примитивен.
    
    $x^5+x+1$ не имеет корней, неприводим. В соответствии со случаем 1, так как период многочлена степени выше 1 должен быть больше 1, $r = q^m -1 = 2^5 - 1 = 31$, тк 31 - простое число. Примитивен.
    
    $x^4+x^2+x+1$ имеет корень 1, приводим. Не примитивен.

    \item $f(x)\in\{x^5+x^2+1, x^3+x^2+x+1, x^4+x^3+x^2+x+1\}, (f(x)\in F_2[x])$
    
    $x^5+x^2+1$ корней нет ($0$ или $1$ не подходят), делители следует искать срели многочленов степени $2,3$, причем неприводимых. Таких 2 степени в нашем поле только одно - $1 + x + x^2$, проверим делимость простым делением уголком - остаток от деления равен $1$, получаем, что $f(x)$ неприводим. В соответствии со случаем 1, так как период многочлена степени выше 1 должен быть больше 1, $r = q^m -1 = 2^5 - 1 = 31$, тк 31 - простое число. Примитивен.

    $x^3+x^2+x+1$ имеет корень 1, приводим. Не примитивен.
    
    $ x^4+x^3+x^2+x+1 $ корней нет ($0$ или $1$ не подходят), делители следует искать срели многочленов степени $2$, причем неприводимых. Таких 2 степени в нашем поле только одно - $1 + x + x^2$, проверим делимость простым делением уголком - остаток от деления равен $x+1$, получаем, что $f(x)$ неприводим. В соответствии со случаем 1, так как период многочлена степени выше 1 должен быть больше 1, период должен быть делителем числа $ q^m -1 = 2^4 - 1 = 15$. Рассмотрим числа $\{3, 5, 15\}$. $x^3 \mod x^4+x^3+x^2+x+1 = x^3 \neq 1$, $x^5 \mod x^4+x^3+x^2+x+1 = x(x^3+x^2+x+1) = 1$, следовательно $r= 5$. Не примитивен.
    
    \item $f(x)\in\{x-2, x-1, x^2+1\}, (f(x)\in Z_3[x])$
    
    Многочлен $x-2$ первой степени, поэтому он неприводим, но тогда период многочлена может быть равен 1. $x - 1 (\mod{x - 2}) = 1 \neq 0$, следовательно период $r \neq 1$. Таким образом, тогда в соответствии со случаем 1 период многочлена $r = q^m - 1 = 3^1 -1 = 2$. Примитивен.
    
    Многочлен $x-1$ первой степени, поэтому он неприводим, но тогда период многочлена может быть равен 1. $x - 1 (\mod{x - 1}) = 0$, следовательно период $r = 1 \neq q^m - 1 = 3^1 -1 = 2$. Не примитивен.
    
    Так как корней у многочлена $x^2+1$ нет ($1,2,3$ не подходят), он неприводим. Период многочлена степени выше 1 должен быть больше 1, тогда в соответствии со случаем 1 период многочлена должен быть делителем числа $q^m -1 = 3^2 -1 = 8$. Рассмотрим числа $\{2, 4, 8\}$. $x^2 \mod x^2 + 1 = -1 \neq 1$,  $x^4 \mod x^2 + 1 = (-1)^2 = 1$. Следовательно, $r=4 \neq q^m -1 = 3^2 -1 = 8$. Не примитивен.

    \item $f(x)\in\{x+2, x^2+2, x^3+2\}, (f(x)\in Z_3[x])$
    
     Многочлен $x+2 = x-1$ первой степени, поэтому он неприводим, но тогда период многочлена может быть равен 1. $x - 1 (\mod{x - 1}) = 0$, следовательно период $r = 1 \neq q^m - 1 = 3^1 -1 = 2$. Не примитивен.
     
     Многочлен $ x^2+2$ имеет корень 1, приводим. Не примитивен.
     
     Многочлен $ x^3+2$ имеет корень 1, приводим. Не примитивен.

    \item $f(x)\in\{x^2+x+1, x^2-x+1, x^2+x+2\}, (f(x)\in Z_3[x])$
    
    Многочлен $ x^2+x+1$ имеет корень 1, приводим. Не примитивен.
    
    Многочлен $ x^2-x+1$ имеет корень 2, приводим. Не примитивен.
    
    Так как корней у многочлена $x^2+x+2$ нет ($1,2,3$ не подходят), он неприводим. Период многочлена степени выше 1 должен быть больше 1, тогда в соответствии со случаем 1 период многочлена должен быть делителем числа $q^m -1 = 3^2 -1 = 8$. Рассмотрим числа $\{2, 4, 8\}$. $x^2 \mod x^2 + x +2 = - x - 2 \neq 1$,  $x^4 \mod x^2 + x +2 = (2x + 1)^2 = x^2 + x + 1 = - 1 \neq 1$. Следовательно, $r=8$. Примитивен.

    \item $f(x)\in\{x^2+1, x^2+2, x^2+2x+2\}, (f(x)\in Z_3[x])$
    
    Так как корней у многочлена $x^2+1$ нет ($1,2,3$ не подходят), он неприводим. Период многочлена степени выше 1 должен быть больше 1, тогда в соответствии со случаем 1 период многочлена должен быть делителем числа $q^m -1 = 3^2 -1 = 8$. Рассмотрим числа $\{2, 4, 8\}$. $x^2 \mod x^2 + 1 = -1 \neq 1$,  $x^4 \mod x^2 + 1 = (-1)^2 = 1$. Следовательно, $r=4\neq q^m -1 = 3^2 -1 = 8$. Не примитивен.
    
    Многочлен $ x^2+2$ имеет корень 1, приводим. Не примитивен.
    
    Так как корней у многочлена $x^2+2x+2$ нет ($1,2,3$ не подходят), он неприводим. Период многочлена степени выше 1 должен быть больше 1, тогда в соответствии со случаем 1 период многочлена должен быть делителем числа $q^m -1 = 3^2 -1 = 8$. Рассмотрим числа $\{2, 4, 8\}$. $x^2 \mod x^2 + 2x +2 = x +1 \neq 1$,  $x^4 \mod x^2 + 2x +2 = (x +1)^2 = x^2 + 2x + 1 = - 1 \neq 1$. Следовательно, $r=8$. Примитивен.

    \item $f(x)\in\{x^2+2x+1, x^2+1, x^2-1\}, (f(x)\in Z_3[x])$
    
     Многочлен $ x^2+2x+1$ имеет корень 2, приводим. Не примитивен.
     
     Так как корней у многочлена $x^2+1$ нет ($1,2,3$ не подходят), он неприводим. Период многочлена степени выше 1 должен быть больше 1, тогда в соответствии со случаем 1 период многочлена должен быть делителем числа $q^m -1 = 3^2 -1 = 8$. Рассмотрим числа $\{2, 4, 8\}$. $x^2 \mod x^2 + 1 = -1 \neq 1$,  $x^4 \mod x^2 + 1 = (-1)^2 = 1$. Следовательно, $r=4\neq q^m -1 = 3^2 -1 = 8$. Не примитивен.
     
      Многочлен $ x^2-1$ имеет корень 1, приводим. Не примитивен.
     
\end{enumerate}

\subsection{Задача 9}

Последовательность $c(1),c(2),c(3),c(4),c(5), ... = 1,2,3,0,4,...$ является суммой двух чисто периодических последовательностей над полем $Z_5$, периоды которых равны $8$ и $12$. Найти $c(51), c(52), c(53), c(50), c(49), c(29), c(28), c(27), c(26), c(25)$.

Одним из периодов суммы двух чисто периодических последовательностей является наименьшее общее кратное периодов двух этих последовательностей, но может быть и меньше. Для решения данной задачи нам достаточно найти какой-либо период $r = $НОК$(8, 12) = 24$.
Тогда $c(51)=c(3)=3, c(52)=c(4)=0, c(53)=c(5)=4, c(50)=c(2)=2, c(49)=c(1)=1, c(29)=c(5)=4, c(28)=c(4)=0, c(27)=c(3)=3, c(26)=c(2)=2, c(25)=c(1)=1$

\subsection{Задача 10}

Укажите все двоичные характеристические многочлены $f\in\{f_1, f_2, f_3\}$, для которых в семействе $S(f)$ существует ЛРП с минимальным периодом $w$.

п/п w Набор функций

\begin{enumerate}
    \item $w=2, f\in \{x+1, x^2+1, x^3+1\}$
    Семейство ЛРП строится на основании характеристической функции $f_i$, минимальный период ЛРП является периодом ее минимального многочена. Минимальный многочлен - некоторый делитель характеристической функции. $f_1$ - неприводимый многочлен, его делители $1, x+1$, тогда возможный период только $w=1$, следовательно $f_1$ - не включаем в ответ. $f_2$ раскладывется как $(x+1)^2$, его делители $1, x+1, x^2+1$, тогда возможные периоды $1, 2$, следовательно $f_2$ - включаем в ответ. $f_3$ раскладывется как $(x+1)(x^2+x+1)$, его делители $1, x+1, x^2+x+1, x^3+1$, тогда возможные периоды $1, 3$, следовательно $f_3$ - не включаем в ответ. 
    
    \item $w=3, f\in \{(x+1)^2, x^3+1, x^3+x+1\}$
    Семейство ЛРП строится на основании характеристической функции $f_i$, минимальный период ЛРП является периодом ее минимального многочена. Минимальный многочлен - некоторый делитель характеристической функции. $f_1$ раскладывется как $(x+1)^2$, его делители $1, x+1, x^2+1$, тогда возможные периоды $1, 2$, следовательно $f_1$ - не включаем в ответ. $f_2$ раскладывется как $(x+1)(x^2+x+1)$, его делители $1, x+1, x^2+x+1, x^3+1$, тогда возможные периоды $1, 3$, следовательно $f_2$ - включаем в ответ. $f_3$ - неприводимый многочлен, его делители $1, x^3+x+1$, тогда возможные периоды $1, 2^3-1=7$, следовательно $f_3$ - не включаем в ответ. 
    
    \item $w=4, f\in \{(x+1)^2, (x+1)^3, x^3+x^2+1\}$
    Семейство ЛРП строится на основании характеристической функции $f_i$, минимальный период ЛРП является периодом ее минимального многочена. Минимальный многочлен - некоторый делитель характеристической функции. $f_1$ раскладывется как $(x+1)^2$, его делители $1, x+1, x^2+1$, тогда возможные периоды $1, 2$, следовательно $f_1$ - не включаем в ответ. $f_2$ раскладывется как $(x+1)^3$, его делители $1, x+1, x^2+1, x^3+x^2+x+1$, тогда возможные периоды $1, 2, 4$, следовательно $f_2$ - включаем в ответ. $f_3$ - неприводимый многочлен, его делители $1, x^3+x^2+1$, тогда возможные периоды $1, 2^3-1=7$, следовательно $f_3$ - не включаем в ответ. 
    
    \item $w=6, f\in \{(x+1)^2, (x+1), x^4+x^2+1\}$
    Семейство ЛРП строится на основании характеристической функции $f_i$, минимальный период ЛРП является периодом ее минимального многочена. Минимальный многочлен - некоторый делитель характеристической функции. $f_1$ раскладывется как $(x+1)^2$, его делители $1, x+1, x^2+1$, тогда возможные периоды $1, 2$, следовательно $f_1$ - не включаем в ответ. $f_2$ - неприводимый многочлен, его делители $1, x+1$, тогда возможный период только $w=1$, следовательно $f_2$ - не включаем в ответ. $f_3$ раскладывется как $(x^2+x+1)^2$, его делители $1, x^2+x+1, x^4+x^2+1$, тогда возможные периоды $1, 2^2-1=3, 2^4-1=15$,, следовательно $f_3$ - не включаем в ответ. 
    
    \item $w=4, f\in \{(x+1)^4, x^3+1, x^2+x+1\}$
    Семейство ЛРП строится на основании характеристической функции $f_i$, минимальный период ЛРП является периодом ее минимального многочена. Минимальный многочлен - некоторый делитель характеристической функции. $f_1$ раскладывется как $(x+1)^4$, его делители $1, x+1, x^2+1, x^3 +x^2+x + 1, x^4 +1$, тогда возможные периоды $1, 2, 4$, следовательно $f_1$ - включаем в ответ. $f_2$ раскладывется как $(x+1)(x^2+x+1)$, его делители $1, x+1, x^2+x+1, x^3+1$, тогда возможные периоды $1, 3$, следовательно $f_2$ - не включаем в ответ. $f_3$ - неприводимый многочлен, его делители $1, x^2+x+1$, тогда возможные периоды $1, 2^2-1=3$, следовательно $f_3$ - не включаем в ответ.
    
    \item $w=3, f\in \{(x+1)^3, x^3+1, x^2+x+1\}$
    Семейство ЛРП строится на основании характеристической функции $f_i$, минимальный период ЛРП является периодом ее минимального многочена. Минимальный многочлен - некоторый делитель характеристической функции. $f_1$ раскладывется как $(x+1)^3$, его делители $1, x+1, x^2+1, x^3 +x^2+x+ 1$, тогда возможные периоды $1, 2, 4$, следовательно $f_1$ - не включаем в ответ. $f_2$ раскладывется как $(x+1)(x^2+x+1)$, его делители $1, x+1, x^2+x+1, x^3+1$, тогда возможные периоды $1, 3$, следовательно $f_2$ - включаем в ответ. $f_3$ - неприводимый многочлен, его делители $1, x^2+x+1$, тогда возможные периоды $1, 2^2-1=3$, следовательно $f_3$ - включаем в ответ.
    
    \item $w=4, f\in \{(x+1)^2, (x+1)^3, (x+1)^4\}$
    Семейство ЛРП строится на основании характеристической функции $f_i$, минимальный период ЛРП является периодом ее минимального многочена. Минимальный многочлен - некоторый делитель характеристической функции. $f_1$ раскладывется как $(x+1)^2$, его делители $1, x+1, x^2+1$, тогда возможные периоды $1, 2$, следовательно $f_1$ - не включаем в ответ. $f_2$ раскладывется как $(x+1)^3$, его делители $1, x+1, x^2+1, x^3+x^2+x+1$, тогда возможные периоды $1, 2, 4$, следовательно $f_2$ - включаем в ответ. $f_3$ раскладывется как $(x+1)^4$, его делители $1, x+1, x^2+1, x^3 +x^2+x+ 1, x^4 +1$, тогда возможные периоды $1, 2, 4$, следовательно $f_3$ - включаем в ответ.
    
    \item $w=6, f\in \{(x^3+1)(x+1)^2, (x^3+1)(x+1)\}$
    Семейство ЛРП строится на основании характеристической функции $f_i$, минимальный период ЛРП является периодом ее минимального многочена. Минимальный многочлен - некоторый делитель характеристической функции. $f_1$ раскладывется как $(x^2+x+1)(x+1)^3$, его делители $1, x+1, x^2+1, x^2+x+1, (x+1)^3, (x^2+x+1)(x+1)^3$, тогда возможные периоды $1, 2, 3, 4, $НОК$(3,4)=12$, следовательно $f_1$ - не включаем в ответ. $f_2$ раскладывется как $(x^2+x+1)(x+1)^2$, его делители $1, x+1, x^2+1, x^2+x+1, (x^2+x+1)(x+1)^2$, тогда возможные периоды $1, 2, 3, $НОК$(2,3)=6$, следовательно $f_2$ - включаем в ответ.
    
    \item $w=8, f\in \{(x+1)^8, (x+1)^7, (x+1)^6\}$
    Семейство ЛРП строится на основании характеристической функции $f_i$, минимальный период ЛРП является периодом ее минимального многочена. Минимальный многочлен - некоторый делитель характеристической функции. $f_1$ раскладывется как $(x+1)^8$, его делители $1, x+1, x^2+1, x^3+x^2+x+1, x^4+1, x^5+x^4+x+1, x^6+x^4+x^2+1, x^7+x^6+x^5+x^4+x^3+x^2+x+1, x^8+1$, тогда возможные периоды $1, 2, 4, 8$, следовательно $f_1$ - включаем в ответ. $f_2$ раскладывется как $(x+1)^7$, его делители $1, x+1, x^2+1, x^3+x^2+x+1, x^4+1, x^5+x^4+x+1, x^6+x^4+x^2+1, x^7+x^6+x^5+x^4+x^3+x^2+x+1$, тогда возможные периоды $1, 2, 4, 8$, следовательно $f_2$ - включаем в ответ. $f_3$ раскладывется как $(x+1)^6$, его делители $1, x+1, x^2+1, x^3+x^2+x+1, x^4+1, x^5+x^4+x+1, x^6+x^4+x^2+1$, тогда возможные периоды $1, 2, 4, 8$, следовательно $f_3$ - включаем в ответ.
    
    \item $w=8, f\in \{(x+1)^5, (x+1)^6, (x+1)^7\}$
    Семейство ЛРП строится на основании характеристической функции $f_i$, минимальный период ЛРП является периодом ее минимального многочена. Минимальный многочлен - некоторый делитель характеристической функции. $f_1$ раскладывется как $(x+1)^5$, его делители $1, x+1, x^2+1, x^3+x^2+x+1, x^4+1, x^5+x^4+x+1$, тогда возможные периоды $1, 2, 4, 8$, следовательно $f_1$ - включаем в ответ. $f_2$ раскладывется как $(x+1)^6$, его делители $1, x+1, x^2+1, x^3+x^2+x+1, x^4+1, x^5+x^4+x+1, x^6+x^4+x^2+1$, тогда возможные периоды $1, 2, 4, 8$, следовательно $f_2$ - включаем в ответ. $f_3$ раскладывется как $(x+1)^6$, его делители $1, x+1, x^2+1, x^3+x^2+x+1, x^4+1, x^5+x^4+x+1, x^6+x^4+x^2+1, x^7+x^6+x^5+x^4+x^3+x^2+x+1$, тогда возможные периоды $1, 2, 4, 8$, следовательно $f_3$ - включаем в ответ.
\end{enumerate}
\backmatter %% Здесь заканчивается нумерованная часть документа и начинаются ссылки

%\include{81-biblio}


%\appendix   % Тут идут приложения

%\include{90-appendix1}

%\include{91-appendix2}

\end{document}

%%% Local Variables:
%%% mode: latex
%%% TeX-master: t
%%% End:
