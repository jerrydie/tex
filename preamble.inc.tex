\sloppy

% Настройки стиля ГОСТ 7-32
\EqInChapter % формулы будут нумероваться в пределах раздела
\TableInChapter % таблицы будут нумероваться в пределах раздела
\PicInChapter % рисунки будут нумероваться в пределах раздела

% Добавляем гипертекстовое оглавление в PDF
\usepackage[
bookmarks=true, colorlinks=true, unicode=true,
urlcolor=black,linkcolor=black, anchorcolor=black,
citecolor=black, menucolor=black, filecolor=black, urlcolor=blue,
]{hyperref}
\AfterHyperrefFix

\usepackage{microtype}% пакет для микротипографии, под pdflatex хорошо улучшает читаемость

% Тире могут быть невидимы в Adobe Reader
\ifInvisibleDashes
\MakeDashesBold
\fi

\usepackage{graphicx}   % Пакет для включения рисунков

\geometry{right=20mm}
\geometry{left=30mm}
\geometry{bottom=20mm}
\geometry{ignorefoot}


% Пакет Tikz
\usepackage{tikz}
\usetikzlibrary{arrows,positioning,shadows}

% Произвольная нумерация списков
\usepackage{enumerate}

% ячейки в несколько строчек
\usepackage{multirow}

% itemize внутри tabular
\usepackage{paralist,array}

\setlength{\parskip}{1ex} % разрыв между абзацами
\usepackage{blindtext}

\usepackage{newfloat}
\DeclareFloatingEnvironment[
placement={!ht},
name=Equation
]{eqndescNoIndent}
\edef\fixEqndesc{\noexpand\setlength{\noexpand\parindent}{\the\parindent}\noexpand\setlength{\noexpand\parskip}{\the\parskip}}
\newenvironment{eqndesc}[1][!ht]{%
    \begin{eqndescNoIndent}[#1]%
\fixEqndesc%
}
{\end{eqndescNoIndent}}

