\Introduction

 Утечка персональных данных является на данный момент серьезной проблемой, с которой в последнее время люди стали сталкиваться все чаще, так как их можно использовать в различным мошеннических схемах - номера кредитных карт, страховых полисов, в связи с чем остро встала задача безопасного хранения подобной информации. Данная ситуация увеличила важность шифрования с сохранения формата данных (format-preserving encryption), так как оно лучше подходит для хранения в базах данных, чем традиционные механизмы с использованием симметричного шифрования. Однако несмотря на исследования вокруг существующих алгоритмов и их постоянную модернизацию находятся способы показать их практическую нестойкость с помощью известных методов, таких как линейный и дифференциальный криптографический анализ, которые с небольшими доработками и уточнениями дешифруют шифртексты, использующие актуальные схемы типа FPE. В данной курсовой работе демонстрируется применение линейного метода анализа схем FPE с настройками на основе сети Фейстеля, а именно стандарта FEA-1, применение линейного метода в российской традиции и его сравнение с представленным в анализируемой статье, а также результаты эксперимента по нахождению линейного статистического аналога для входных и выходных последовательностей шифра.
