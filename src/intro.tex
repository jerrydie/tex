\Introduction

 С ускорением глобальной информатизации все острее встает вопрос о защите информации, в частности персональных данных. Несмотря на то, что существуют законы, регламентирующие порядок хранения и обработки персональных данных, возлагающие ответственность за их сохранность на операторов персональных данных, в большинстве случаев эта информация хранятся в базах в открытом виде, и несанкционированный доступ к ней не требует больших усилий от злоумышленника. В связи с тем, что последствия реализации данного типа угроз могут быть достаточно серьезными, остро встает задача безопасного хранения подобных данных. Для персональной информации наиболее подходящим способ защиты является шифрование с сохранение формата (format-preserving encryption, FPE), так как в отличие от традиционных механизмов шифрования, оно, во-первых, позволяет программам, обрабатывающим данные как переменные заранее заданного типа, так же успешно обрабатывать и зашифрованные данные, и, во-вторых, позволяет скрыть сам факт шифрования. 
 В 2021 году  Тим Бейн, аспирант Лёвенсокго католического университета в Бельгии, представил работу \cite{main_paper}, в которой продемонстрировал, как можно уменьшить сложность атак на FPE-алгоритмы с настройками с помощью линейного криптографического анализа. В данной курсовой работе демонстрируются: описание линейного метода анализа схем FPE с настройками на основе сети Фейстеля, а именно стандарта FEA-1; применение линейного метода с акцентом на использование статистических критериев с использованием теоретических обоснований, представленных в анализируемой статье; а также результаты эксперимента по нахождению линейного статистического аналога для входных и выходных последовательностей шифропреобразования.
