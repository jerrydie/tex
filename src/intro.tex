\Introduction

 С ускорением глобальной информатизации все острее встает вопрос о защите данных, в частности персональных данных. Несмотря на то, что существуют законы, регламентирующие порядок хранения и обработки персональных данных, возлагающие ответственность за их сохранность на операторов персональных данных, в большинстве случаев эти данные хранятся в базах в открытом виде, и несанкционированный доступ к ним не требует больших усилий от злоумышленника. В связи с тем, что последствия реализации данного типа угроз могут быть достаточно серьезными, остро встает задача безопасного хранения. Для информации персонального типа наиболее подходящим способ защиты является шифрование с сохранение формата (format-preserving encryption, FPE), так как оно лучше подходит для хранения в базах данных, чем традиционные механизмы с использованием симметричного шифрования. Однако несмотря на исследования вокруг существующих алгоритмов и их постоянную модернизацию, находятся способы показать их практическую нестойкость с помощью известных криптографических методов, таких как линейный криптоанализ, который с небольшими доработками и уточнениями может позволить дешифровать данные. В данной курсовой работе демонстрируются: описание линейного метода анализа схем FPE с настройками на основе сети Фейстеля, а именно стандарта FEA-1; применение линейного метода в российской традиции и его сравнение с подходом, представленным в анализируемой статье; а также результаты эксперимента по нахождению линейного статистического аналога для входных и выходных последовательностей шифропреобразования.
